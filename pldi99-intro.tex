% -*- latex -*- This is a LaTeX document.
% $Id: pldi99-intro.tex,v 1.6 1999-11-13 05:56:07 rinard Exp $
%%%%%%%%%%%%%%%%%%%%%%%%%%%%%%%%%%%%%%%%
\section{Introduction}

Static Single Assignment (SSA) form transforms the
program so that exactly one definition of each 
variable reaches each use. Traditional dataflow
analyses, which propagate information from variable
definitions to uses, therefore become much simpler
to express. Instead of generating an analysis 
result for each variable at each point in the
program, SSA allows the analysis to generate a 
result for each variable. This sparse
representation improves both the simplicity
and the efficiency of the analysis. 

But definitions are not the only place where
an analysis can extract information about the
values of variables. Conditional branches also
provide information about the values of variables.
Consider what happens when one attempts to incorporate
this information into an SSA-based analysis. The
original problem that SSA eliminated (the need to
extract information for each variable at each
program point) returns. Variable names do not
change at the conditional branch, even though
the compiler has different information along the
two control-flow paths.

The resulting mismatch between variable names
and dataflow information also produces an efficiency
problem. Instead of propagating information
directly from definitions to uses, the analysis
must propagate the information through all 
program points, whether the program point uses 
the information or not. 

Inspired by this observation, we have developed
a new program representation, 
Static Single Information (SSI) form,  that recaptures
the advantages of SSA form for predicated analyses,
or analyses which use the predicates in conditional
branches to extract analysis information.
The insight behind this representation is the use
of \sigfunction{}s, which produce new names
for variables at splits in the control flow.
The analysis can then associate information
with variable names, and propagate
the information efficiently and directly from 
information definition points to uses. 

In addition to these practical properties, 
SSI form has several appealing 
theoretical properties. It is always possible 
to place \sigfunction{}s so that the
number of \sigfunction{}s is linear in the size
of the original input program. Furthermore, our
placement algorithm also runs in linear time. 
Finally, and perhaps most importantly, SSI form
allows us to recast compound dataflow analyses as
a flat, unified system of constraints. This
formulation allows us to generalize the standard
fixed-point solution
mechanism for dataflow equations to include 
constraint resolution rules. The result is 
a more uniform and powerful analysis framework.

We have implemented a compiler infrastructure for Java,
the MIT Flex system, that uses SSI form \cite{flexweb}.
We have used this compiler infrastructure to implement several 
predicate-based analyses, including an analysis that
detects redundant array bounds and null reference checks, an
analysis that determines the number of bits
required to represent values in different variables,
and a conditional constant propagation analysis.

Our experimental results show that our analyses
execute efficiently and extract information that can
be used to significantly optimize the program.
Furthermore, we believe that SSI form significantly
simplified the implementation of these analyses.
Our experiences have led us to use SSI form 
as the primary program representation in the Flex
compiler system.
