% -*- latex -*- This is a LaTeX document.
% $Id: exec.tex,v 1.3 1999-04-29 18:45:57 cananian Exp $
%%%%%%%%%%%%%%%%%%%%%%%%%%%%%%%%%%%%%%%%%%%%%%%%%%
\documentclass[12pt,notitlepage,twoside]{article}

% A breaking non-space for URLs
\newcommand{\bnsp}{\hspace{0pt}}
% environment for code examples
\newenvironment{samplecode}[1][1]
  {\setlength{\tabcolsep}{2\tabcolsep}%
\renewcommand{\>}{~~~~}%
\newcommand{\comment}[1]{\rm\it // ##1}%
\newcommand{\com}[1]{\comment{##1}}%
\newcommand{\ells}[1]{\ifcase#1\or l\or l|l\or l|l|l\or l|l|l|l\or
l|l|l|l|l\or l|l|l|l|l|l\fi}%
\begin{center}\tt\begin{tabular}{\ells{#1}}}
  {\end{tabular}\end{center}}

% phi and sigma functions
\newcommand{\phifunction}{$\phi$-function}
\newcommand{\sigfunction}{$\sigma$-function}
\newcommand{\xifunction}{$\xi$-function}
\newcommand{\tuple}[1]{\ensuremath{\left\langle #1 \right\rangle}}
\newcommand{\xivec}[2]{\ensuremath{\left[\tuple{#1}\atop\tuple{#2}\right]}}

% Cytron's path symbol
\newcommand{\path}{\ensuremath{\rightarrow}}
\newcommand{\pathplus}{\ensuremath{\stackrel{+}{\path}}}

% SSI0/+
\newcommand{\ssizero}{SSI$_0$}
\newcommand{\ssiplus}{SSI$^+$}

\title{Executability in the Static Single Information Form}
\author{C. Scott Ananian}
\date{\today \\ $ $Revision: 1.3 $ $}

\begin{document}
\bibliographystyle{plain}

\maketitle

The Static Single Information (SSI) form, as originally presented,
requires control-flow graph information in order to be executable.
Executability is desirable because it enables the use of abstract
interpretation when designing algorithms \cite{pingali90:dfg}.  We
would like to have a demand-driven operational semantics for SSI form
that does not require control-flow information; thus freeing us to
more flexibly reorder execution.

In particular, we would like a representation that eliminates
unnecessary control dependencies such as exist in the program of
figure \ref{fig:ctrldep}.  A control-flow graph for this program, as
it is written, will explicitly specify that no assignments to
\texttt{B[]} will take place until all elements of \texttt{A[]} have
been assigned; that is, the second loop will be
\textit{control-dependent} on the first.  We would like to remove this
control dependence in order to provide greater parallelism---in this
case, to allow the assignments to \texttt{A[]} and \texttt{B[]} to
take place in parallel, if possible.

\begin{figure}[t]
\begin{samplecode}
for (int i=0; i<10; i++)\\
\>A[i] = x;\\
for (int j=0; j<10; j++)\\
\>B[j] = y;\\
\end{samplecode}
\caption{An example of unnecessary control dependence: the second loop
is \textit{control-dependent} on the first and so assignments to
\texttt{A[]} and \texttt{B[]} cannot take place in parallel.}
\label{fig:ctrldep}
\end{figure}

The modifications outlined here will extend SSI form in order to
provide a useful and descriptive operational semantics.  We will call
the extended form \ssiplus.  For clarity, SSI form as originally
presented we will call \ssizero.

\section{SSI extensions}
Although a demand-driven execution model can be constructed for
\ssizero,  it fails to handle loops and imperative
constructs well. \ssiplus\ form addresses these deficiencies.

\subsection{Imperative constructs and pointer variables}
The presentation of \ssizero\ ignored pointers,
concentrating on so-called register variables.  Extending \ssizero\ to
handle these imperative constructs is quite easy: we simply define a
``variable'' $S$ to represent an updatable store.  This variable is
renamed and numbered as before, so that $S_0$ represents the initial
contents of the store and $S_i, i>0$ represents the contents of the
store after some sequence of writes.  Figure \ref{fig:store} shows a
simple imperative program in \ssiplus\ form.  Note that modifications
to the store typically take the previous contents of the store as
input, and that subroutines with side-effects modifying the store must
be written in \ssiplus\ form such that they both take a store and
return a store.

\begin{figure}[t]
\begin{samplecode}[2]
\com{swap A[i] and B[j]} & \com{\ssiplus\ form:}\\
x = A[i];	& $x_0$ = FETCH($S_0$, $A_0 + i_0$) \\
y = B[j];	& $y_0$ = FETCH($S_0$, $B_0 + j_0$) \\
A[i] = y;	& $S_1$ = STORE($S_0$, $A_0 + i_0$, $y_0$); \\
B[j] = x;	& $S_2$ = STORE($S_1$, $B_0 + j_0$, $x_0$); \\
\end{samplecode}
\caption{Use of the ``store variable'' $S_x$ in \ssiplus\ form.}
\label{fig:store}
\end{figure}

For simplicity, most of the remaining examples will not address the store.

\subsection{Loop constructs}
\begin{figure}[t]
\begin{samplecode}[3]
\com{a simple loop} & \com{\ssizero\ form:} & \com{\ssiplus\ form:}\\
j=1;	& \>$j_0 = 1$			& $j_0=1$\\
i=0;	& \>$i_0 = 0$			& $i_0=0$\\
do	& L1:			& $\xivec{j_1}{i_5}=\xi(\xivec{j_0}{i_3})$\\
$\{$	& \>$i_1 = \phi(i_0, i_3)$	& $i_1 = \phi(i_0, i_5)$ \\
\>i+=j; & \>$i_2 = i_1 + j_0$		& $i_2 = i_1 + j_1$ \\
$\}$ while (i<5);	& \>$P_0=(i_2<5)$	& $P_0=(i_2<5)$\\
	& \>if $P_0$ goto L1		& $\tuple{i_3,i_4}=\sigma(P_0, i_2)$ \\
	& \>~~$\tuple{i_3,i_4}=\sigma(i_2)$ &\\
\end{samplecode}
\caption{A simple loop, in \ssizero\ and \ssiplus\ forms.}
\label{fig:loop}
\end{figure}

The center column of figure \ref{fig:loop} shows a typical loop in
\ssizero\ form.  Note first that an explicit ``control flow''
expression (\texttt{goto L1}) is required in order to make sense of
the program.  Note also that $i_1$, $i_2$ and $i_3$ are potentially
\textit{dynamically} assigned many times, although \textit{statically}
they have only one definition each.  This complicates any sort of
demand-driven semantics: should the \phifunction{} demand the value of
$i_0$, or $i_3$, when it is evaluated the first time?  Which of the
values of $i_3$ does it receive when the \phifunction{} is
subsequently evaluated?  A token-based dataflow interpretation fails
as well: it is easy to see that tokens for $i_x$ flow around the loop
before flowing out at the end, but the token for $j_0$ seems to be
``used up'' in the first iteration.

\ssiplus\ introduces a \xifunction\ in the block of \phifunction{s} to
clarify the loop semantics. The left-hand column of figure
\ref{fig:loop} illustrates the nature of this function.  The
\xifunction{} arbitrates loop iteration, and will be defined precisely
in the operational semantics of \ssiplus\ form.  For now note that it
relates iteration variables (the top tuple of the parameter and result
vectors) to loop invariants (the bottom tuple of the vectors).  We
followed the statement ordering of \ssizero\ in the figure, but unlike
\ssizero, the statements of \ssiplus\ could appear in any order
without affecting their meaning---and so the statement label
\texttt{L1} of the \ssizero\ representation is unnecessary in \ssiplus.

There is at most one \xifunction{} per \phifunction{} block, and it
always precedes the \phifunction{s}.  Construction of \xifunction{s}
takes place before the renaming step associated with SSI form, and the
\xifunction{s} are then renamed in the same manner as any other
definition.  The top tuple of the constructed \xifunction{} contains
the names of all variables reaching the guarded \phifunction{} via a
backedge, and the bottom tuple contains all variables used inside the
guarded loop that are {\em not} mentioned in the header's
\phifunction{}.\footnote{I can give you a rigorous algorithm: the top
tuple is found with a depth-first search of the source
control-flow graph, and the bottom tuple can be determined by a
traversal of the same SESE-region tree which generates \ssizero\
form.}

\section{Operational Semantics}
We will base the operational semantics of \ssiplus\ on a demand-driven
dataflow model.  We will define both a cycle-oriented semantics and an
event-driven semantics, which (incidentally) correspond to synchronous
and asynchronous hardware models.

\subsection{Cycle-oriented semantics}

\begin{figure}[t]
\begin{center}\begin{tabular}{lc}
$t =\phi(t_1, t_2)$: & foo\\
\end{tabular}\end{center}
\caption{Cycle-oriented transition rules for \ssiplus}
\label{fig:cyclesemantics}
\end{figure}

\subsection{Event-driven semantics}

\begin{figure}[t]
\begin{center}\begin{displaymath}\begin{array}{lc}
t =\phi(t_1, t_2): &
\frac{t_2\mbox{ defined }\Delta t,t_1\mbox{ undefined}}{bar}\\
\end{array}\end{displaymath}\end{center}
\caption{Event-driven transition rules for \ssiplus}
\label{fig:eventsemantics}
\end{figure}

\section{Hardware correspondence}

\bibliography{harpoon}
\appendix
\end{document}
