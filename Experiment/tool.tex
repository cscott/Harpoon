\documentclass{article}
\usepackage{amsmath}
\usepackage{amssymb}
\usepackage{dsfont}
\usepackage{epsfig}
\usepackage{bbold}

\title{}
\author{Debugging experiment - The Tool}
\date{July 30, 2003}

\setlength{\topmargin}{0in}
\setlength{\oddsidemargin}{0in}
\setlength{\evensidemargin}{0in}
\setlength{\textwidth}{6.6in}
\setlength{\textheight}{9in}
\setlength{\parindent}{0.5in}



\begin{document}
\maketitle
\begin{flushleft}

\thispagestyle{empty}

\vspace{0.5in}

As already mentioned in the ``Consistency Constraints'' document, our tool 
receives a set of consistency constraints and checks if any of these 
constraints is violated.  The consistency constraints for the file system
benchmarks are the following:

\begin{verbatim}
[],sizeof(SuperBlock)=1
[],sizeof(GroupBlock)=1
[],sizeof(InodeTableBlock)=1
[],sizeof(InodeBitmapBlock)=1
[],sizeof(BlockBitmapBlock)=1
[],sizeof(RootDirectoryInode)=1

[forall u in UsedInode], u.inodestatus=Used
[forall f in FreeInode], f.inodestatus=Free

[forall u in UsedBlock], u.blockstatus=Used
[forall f in FreeBlock], f.blockstatus=Free

[forall i in UsedInode], i.referencecount=sizeof(i.~inodeof) or i in RootDirectoryInode
\end{verbatim}

Each consistency constraint is composed of a quantifier $Q$, followed by a 
body $B$.  The quantifier $Q$ can be either empty (``[]''), or of the form 
``forall $e$ in $S$'', where $S$ is a set and $e$ an element of $S$.
The body $B$ contains a number of basic statements, which are combined by
the logical operators $\{not, and, or\}$.  These statements must hold for  
all the possible values of the quantifier $Q$, or simply hold if $Q$ 
is empty.  Finally, the only notation you might not understand is 
``sizeof($S$)'', which denotes the number of elements of the set $S$.

\vspace{0.1in}

The tool checks whether any of these constraints is violated.  If so, the tool 
provides information about the first violated constraint and then terminates 
the program.  The tool specifies what constraint is violated and for what 
values of the quantifier, and also provides the current values of the elements 
appearing in the violated constraint.
The information is provided in the following form:

\begin{verbatim}
Constraint violation
   Violated constraint: [forall u in UsedBlock], u.blockstatus=Used
   State for which it was violated: u=29
   Actual value(s): u.blockstatus=Free
\end{verbatim}

All you need to use in order to understand the message above are the 
set and function descriptions included in the ``Consistency Constraints'' 
document. 
Using those descriptions, the message above signals that the block {\it u=29}, 
which belongs to the set {\it UsedBlock} -- 
which means that block {\it 29} is actually in use (there exists an inode 
pointing to it or is one of the special blocks {\it SuperBlock}, 
{\it GroupBlock}, {\it InodeTableBlock}, {\it BlockBitmapBlock}) 
-- was incorrectly marked in the block bitmap as free instead of used 
(because {\it u.blockstatus} has the value {\it Free} instead of {\it Used}).


\vspace{0.1in}
In order to call the tool, you have to use the procedure 
{\it calltool(char *text)}, which prints the text {\it text} and then 
calls the tool.   For example, if you want to call the tool after a new file
was open, you can add after
\begin{verbatim}
int fd = openfile(ptr,filename);
\end{verbatim}

the following lines of code:
\begin{verbatim}
char text[100]; // an error message used to identify this call
sprintf(text, "After openfile(%s)", filename); // construct the error message
calltool(text); // call the tool
\end{verbatim}

If a constraint violation is found when the $calltool$ procedure
is called, the program terminates and prints the contents of $text$ 
followed by a consistency violation message.
Otherwise, it simply prints the contents of $text$ and continues the execution 
of the program.

\vspace{0.1in}
The tool basically works like an {\it assert} statement. 
You should use the tool as often as possible, in order to localize and 
understand the bug.  However, note that there are points in the program
where it is ``normal'' for the constraints not to hold.  For example, 
if a new block $b$ of information is added to a file, we need to modify the 
inode corresponding to that file to point to $b$, and we have to mark $b$ as
used in the block bitmap. Of course, the constraints should hold before and
after these two operations, but it is ``normal'' not to hold  between them.


\vspace{0.1in}
Finally, note that there exists bugs that don't corrupt the data structures, 
and which consequently don't violate the constraints.   Thus, the tool can
be extremely helpful for some bugs, and totally useless for others. However,
try to make as much use of the tool as possible.


\end{flushleft}
\end{document}
