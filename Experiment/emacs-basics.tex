\documentclass{article}

\usepackage{amsmath}
\usepackage{amssymb}
\usepackage{dsfont}
\usepackage{epsfig}
\usepackage{bbold}

\title{}
\author{Emacs basics\footnote{Adapted from the document found at http://www.cs.dartmouth.edu/~cs23/emacs-basics.html}}
\date{July 30, 2003}

\setlength{\topmargin}{0in}
\setlength{\oddsidemargin}{0in}
\setlength{\evensidemargin}{0in}
\setlength{\textwidth}{6.6in}
\setlength{\textheight}{9in}
\setlength{\parindent}{0.5in}



\begin{document}
\maketitle
\begin{flushleft}

\thispagestyle{empty}

The operating system under which the experiment is going to be run is a 
UNIX-type operating system.  You can use whatever editor you like during
this experiment.  If you are not familiar with any editor under UNIX, here 
is a quick guide to one of them, called Emacs.

\vspace{0.2in}

Emacs commands usually involve the Control key.  A control key command is 
written with C-. For example, C-x means "hold down the Control key and press 
the x key."

\section*{Starting and quitting Emacs}
To open a certain file in Emacs, type 'emacs {\it filename} \&' on the command
line.  To simply run Emacs, type 'emacs \&'.

To exit Emacs, type 'C-x C-c'.   If any files have been changed since they were
last saved, Emacs offers to save them. 


\section*{Saving}
To save the current file, use the command 'C-x C-s'.

To save the file under another name (``save as''), use the command  'C-x C-w', 
and enter the new file name when prompted.


\section*{Basics commands to change the text}
To change the current position of the cursor, simply use the arrow keys.

To move the cursor at the beginning of the line, type 'C-a'.

To move the cursor at the end of the line, type 'C-e'.

To delete the character just before the cursor, use the Backspace key.

To delete the character on which the cursor is positioned, use the command 
'C-d'.

To undo changes, use the command 'C-x u'. (type C-x and then press the u key).

To delete the current line, place the cursor at the beginning of the line, 
and then use the command 'C-k'.


\section*{Searching}
To perform an 'incremental' search, type 'C-s'. Emacs will prompt you for a 
search string. As you type characters, they add to the search string and are 
found. Type DEL to cancel characters from the end of search string. Type ESC
to exit, leaving the point at the location found. Type C-s  to search again 
forward, or C-r to search again backward. Typing C-g when the search has failed
transforms the search string back to what has been found successfully. Typing 
C-g when the search is successful aborts and moves the point to the starting 
point.

Note that Emacs has an unusual notion of case sensitivity when it searches for 
text. If the search string contains only lowercase letters, then it matches 
either uppercase or lowercase letters in the file. But if the search string 
contains any uppercase letters, the case must match the file exactly. 


\section*{Getting out of trouble}
If anything goes wrong while you try to perform a certain operation, type 
'C-g' until things are ok.  If this doesn't work, ask for assistance.

\end{flushleft}
\end{document}