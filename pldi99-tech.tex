% -*- latex -*- This is a LaTeX document.
% $Id: pldi99-tech.tex,v 1.12 1999-11-13 07:07:30 cananian Exp $
%%%%%%%%%%%%%%%%%%%%%%%%%%%%%%%%%%%%%%%%
\section{The Static Single Information form}
In this section we will provide a formal specification of SSI form and
its \emph{minimal} and \emph{pruned} variants.  We will also provide
efficient algorithms for constructing these representations.

% Technical Contents (definitions, algorithms, etc)
%  Program Representation
%   Sigmas, placement algorithms, theorems.
%  Constraint extraction and framework.
%  Contraint systems and resolution analysis (for example)
%   Bitwidth
%   Array bounds analysis, etc.

\subsection{Definition of SSI form}
SSI form is an extension of the SSA form introduced in \cite{cytron89:ssa}.
Building SSI form involves adding pseudo-assignments for a variable $V$:
\begin{enumerate}
\item[$(\phi)$] at a control-flow merge when disjoint paths from a
conditional branch come together and at least one of the paths
contains a definition of $V$; and
\item[$(\sigma)$] at locations where control-flow splits and at least
one of the disjoint paths from the split uses the value of $V$.
\end{enumerate}

\subsection{Criteria for inserting \sigfunction{s}}
To minimize the number of \sigfunction{s}, there should be a
\sigfunction{} for variable $a$ at node $z$ of the flowgraph exactly
when:
\begin{enumerate}
\item node $x$ contains a use of $a$,
\item node $y$ contains a use of $a$,
\item there is a nonempty path $P_{zx}$ of edges from $z$ to $x$,
\item there is a nonempty path $P_{zy}$ of edges from $z$ to $y$, and
\item paths $P_{zx}$ and $P_{zy}$ do not have any node in common
except $z$ (that is, $z$ is the point of divergence for these paths).
\end{enumerate}
We will call this the \textit{path-convergence criterion} for
inserting \sigfunction{s}.  We consider the start node to contain an
implicit definition of every variable, and the end node to contain an
implicit use of every variable.

\mycomment{
Note that this criterion is very similar to the path-con\-ver\-gence
criterion for inserting \phifunction{s} described in
\cite{appel:modern,cytron91:ssa}.  And, just as in the \phifunction{}
case, the above criterion is open-ended: since the \sigfunction{}
itself counts as a use of $a$, we must iterate the above definition
to determine the final set of \sigfunction{s}.
}

Upon examination, we see that the path-con\-ver\-gence criteria for $\phi$- and
\sigfunction{s} interact.  Since \sigfunction{s} are variable
definitions and \phifunction{s} are variable uses, the set of
equations defined by the respective criteria must be iterated together 
in order to find the necessary function sets.  The total number of
$\phi$- and \sigfunction{s} remains linear, however: we can only place 
a single $\phi$- and/or \sigfunction{} per variable at any given
flowgraph node, so the total number of added functions is limited to 
$2 \cdot N \cdot V$.

\mycomment{
Cytron et al.\ \cite{cytron91:ssa} has shown that the iterated
path-con\-ver\-gence criterion for \phifunction{s} is equivalent to the
\textit{iterated dominance frontier criterion}: whenever a node $x$
contains a definition of a variable $a$, then any node $z$ in the
dominance frontier of $x$ needs a \phifunction{} for $a$; nodes in the
dominance frontier of any added \phifunction{} similarly need
\phifunction{s}.  The equivalent statement for \sigfunction{s}
references \textit{uses} of the variable $a$ and nodes in its
post-dominance frontier. 
}

\subsection{Variable renaming after \phisigfunction insertion}

Once the compiler has determined where to place the 
\phifunction{s} and \sigfunction{s}, it renames variables to
satisfy the following two properties:
\begin{property}[Naming after \phifunction{s}.]\label{crit_phiname}
For every node $x$ containing a definition of a variable $a$ in
the renamed program and node $y$ containing a use of that variable, there
exists at least one non-empty path $P_{xy}$ of edges from $x$ to $y$
and no such path contains a definition of $a$ other than at $x$.
\end{property}
\begin{property}[Naming after \sigfunction{s}.]\label{crit_signame}
For every pair of nodes $x$ and $y$ containing uses of a
variable $a$ defined at a node $z$ in the renamed program, either
every nonempty path $P_{zx}$ of edges from $z$ to $x$ must contain
node $y$, or every nonempty path $P_{zy}$ of edges from $z$ to $y$
must contain $x$.
\end{property}

In addition, correctness requires that the following hold:
\begin{property}[Correctness.]\label{crit_correct}
Along any possible control-flow path in a program being executed
consider any use of a variable $a$ in the original program and the
corresponding use of $a_i$ in the renamed program.  Then, at every
occurrence of the use on the path, $a$ and $a_i$ have the same value.
The path need not be cycle-free.
\end{property}

\subsection{Minimal and pruned SSI forms}

\emph{Minimal} and \emph{pruned} SSI forms can be defined which
parallel their SSA counterparts.  \emph{Minimal} SSI has
the smallest number of \phisigfunction{s} such that the above
conditions are satisfied.  \emph{Pruned} SSI form is the minimal form
with any unused \phisigfunction{s} deleted; that is, it contains no
\phisigfunction[or]{s} after which there are no subsequent
non-\phisigfunction[or]{} uses of any of the variables defined on the
left-hand side.\footnote{An even more compact SSI form may be produced
by removing \sigfunction{s} for which there are uses for \emph{exactly
one} of the variables on the left-hand side, but by doing so one loses
the ability to perform renaming at some control-flow splits which may generate
additional value information.}  Figure~\vref{fig:prunedssi} compares 
minimal and pruned SSI form for an example program.
\begin{myfigure*}
\begin{center}
\input{Figures/THex1ssi} \vline\ \input{Figures/THex1ssiPr}
\end{center}
\caption[Minimal and pruned SSI forms.]
{Minimal (left) and pruned (right) SSI forms.}
\label{fig:prunedssi}
\end{myfigure*}

Note that, as in SSA form, pruned SSI
does not strictly satisfy the SSI constraints because it omits dead
\phisigfunction{s} otherwise required by the path-convergence criteria.
In practice, a subtractive
definition of pruned form --- generate minimal form and then remove
the unused \phisigfunction{s} --- is most useful, but a constructive
definition can be generated from the standard SSI form definition as
follows:
\begin{enumerate}
\item The convergence/divergence node $z$ of the path-convergence
criteria for inserting \phisigfunction{s}  must also satisfy: ``and
there exists a nonempty path $P_{zu}$ from $z$ to a $u$, a use of $a$ in the
original program, which does not contain another definition of $a$.''
  \label{amend_place}
\item The boundary condition specified by the path-convergence
criterion for the node \code{END} can be
loosened as follows (emphasis indicates modifications):
``For the purposes of this definition, the \code{START} node is
assumed to contain a definition for every variable in the original
program and the \code{END} nodes a use \emph{for every variable live
at \code{END}} in the original program.''
  \label{amend_boundary}
\end{enumerate}

Pruned form is defined as having the minimal set of \phisigfunction{s}
that satisfy the amended conditions.  It can easily be verified that
the modifications suffice to eliminate unused \phisigfunction{s}: if
the variable defined in a \phisigfunction[or]{} is used, there must
exist a nonempty path $P_{zu}$ as mandated by amendment~\ref{amend_place},
where amendment~\ref{amend_boundary} lets $u = \code{END}$ for
variables live exiting the procedure and thus usefully defined.

\begin{property}\label{pty:pruned_live}
A node $Z$ gets a \phisigfunction[or]{} for some variable $V_i$ in
pruned SSI form only if the corresponding variable $V$ is live at $Z$
in the original program.
\end{property}
\begin{proof}
This is a trivial restatement of amendment~\ref{amend_place}.  A
variable $v$ is said to be live at some node $N$ if there exists
a node $U$ using $v$ and a path $N\pathplus U$ on which no definitions
of $v$ are to be found.  If $V$ is not live at $Z$ then no path
$Z\pathplus U$ satisfying the amended path-convergence criteria
can be found and neither a
\phisigfunction[or]{} can be placed.  Amendment~\ref{amend_boundary}
ensures this holds true at boundaries.
\end{proof}

\section{SSI construction algorithms}

Construction of SSI form takes place in two phases.  First, the required
\phisigfunction{s} for each variable are inserted at control-flow
merge and split points.  Then renaming is performed to create a valid
SSI form program.

\subsection{Placement algorithms}
Sreedhar and Gao have shown \cite{sreedhar95:lintime} that it is
possible to place \phifunction{s} in time proportional to the size of
the program.  With appropriate modifications to the algorithm, it can
be used to place \sigfunction{s}.  However, as noted above, $\phi$-
and \sigfunction{} placement is not independent: the placement of
\phifunction{s} necessitates additional \sigfunction{} placement, and
vice versa.  Thus, the (linear time) placement algorithms can be run
iteratively to find a fixed point.  Since the maximum number of
$\phi$- or \sigfunction{s} is proportional to the size of the program,
it is obvious that no more than $N$ iterations will be required,
resulting in a worst-case running time of $O(N^2)$.  In practice
one would expect relatively few iterations\footnote{The number of required
iterations is related to the maximum loop nesting depth, which Knuth
\cite{knuth74:fortran} showed remains small for ``human-generated'' programs.}
yielding a near-linear runtime.
% rewrite last sentence to reference studies on control-flow depth in
% actual programs.

The most common construction algorithm for SSA form
\cite{cytron91:ssa} uses dominance frontiers and suffers from a
possible quadratic blow-up in the size of the dominance frontier for
certain common programming constructs.  Various improved algorithms
use such things as DJ graphs \cite{sreedhar95:lintime} and the
dependence flow graph \cite{johnson93:dfg} to achieve $O(EV)$ time
complexity for \phifunction placement.  We build on this work to
achieve $O(EV)$ construction of SSI form, and present a new algorithm
for variable renaming in SSI form after \phisigfunction{s} are placed.

% refer to program structure tree and cycle-equivalency algorithms.
Our construction algorithm begins with a program structure tree of
single-entry single-exit (SESE) regions, constructed as described by
Johnson, Pearson, and Pingali \cite{johnson94:pst}.  
\mycomment{
We will review
the algorithms involved, as their published descriptions
\cite{johnson93:sese} contain a number of errors.
% copy algorithm 5.3 from thesis here.
}
\mycomment{
\begin{myalgorithm}\small
% algorithm for phi/sigma placement.
\begin{verbatim}
Procedure place (G:CFG)
{
  let r be the top-level region for G;

  for each variable v in G, do
    place(r, v);
}

Procedure place(r:region, v:variable):~boolean
{
  /* Post-order traversal */
  flag:=false;
  for each child region r', do
     flag:=place(r', v) || flag;

  for each node n in region r not contained in a child region, do
     if n contains a use or definition of v, then
        flag := true;

  /* add phis/sigmas to merges/splits where v may be live */
  if flag=true, then {
    for each node n in region r not contained in a child region, do {
      if MaybeLive(n) = true, then {
        if the input arity of n exceeds 1, then
          place a phi function for v at n;
        if the output arity of n exceeds 1, then
          place a sigma function for v at n;
      }
    }
  }

  return flag;
}
\end{verbatim}

\caption{Placing \phisigfunction{s}.}\label{alg:SSIplace}
\end{myalgorithm}
}

We split the construction of SSI form into two parts: placing \phisigfunction{s}
and renaming variables.  The placement algorithm runs in $O(N V_0)$
time, and is presented as Algorithm~\vref{alg:SSIplace}.  
The algorithm is parameterized on
a function called \code{MaybeLive}.  
For minimal SSI form,
\code{MaybeLive} should always return \code{true}.  Faster practical
run-time may be obtained if pruned SSI form is the desired goal by
allowing \code{MaybeLive} to return any conservative approximation of
variable liveness information, which will allow early suppression of
unused \phisigfunction{s}.  Note that \code{MaybeLive} need not be
precise; conservative values will only result in an excess of
\phisigfunction{s}, not an invalid SSI form.  Section
\ref{sec:unusedcode} describes a post-processing algorithm to
efficiently remove the excess \phisigfunction{s}.%
\mycomment{
\footnote{Note that equivalent results could be
obtained by adding a \phifunction{} for every variable at every merge
and a \sigfunction{} for every variable at every split, and
post-processing.  In fact the same time bounds ($O(N V_0)$) would be
obtained.  There is a large practical difference in actual runtime and
space costs, however, which motivates our more efficient approach.}
The remainder of this section will be devoted to a correctness proof
of Algorithm~\ref{alg:SSIplace}.
}

\begin{lemma}\label{lem:sese_child}
No \phifunction{s} (\sigfunction{s}) for a variable $v$ are needed in
an SESE region not containing a definition (use) of $v$.
\end{lemma}
\begin{proof}
See Appendix~\ref{app:proofs}.
\end{proof}
\mycomment{
\begin{proof}
Let us assume a \phifunction for $v$ is needed at some node $Z$
inside an SESE not containing a definition of $v$.  
Then by condition \ref{crit_phiplace} of the SSI
form definition, there exist paths $X \pathplus Z$ and $Y \pathplus Z$
having no nodes but $Z$ in common where $X$ and $Y$ contain either
definitions of $v$ or \phisigfunction[or]{s} for $v$.  Choose any such
paths:
\begin{description}
\item[Case I:] Both $X$ and $Y$ are outside the SESE.  Then, as there
is only one entrance edge into the SESE, the paths $X \pathplus Z$ and
$Y \pathplus Z$ must contain some node in common other than Z.  But
this contradicts our choice of $X$ and $Y$.
\item[Case II:] At least one of $X$ and $Y$ must be inside the SESE.
If both $X$ and $Y$ are not definitions of $v$ but rather
\phisigfunction[or]{s} for $v$, then by recursive application of this proof
there must exist some choice of $X$, $Y$, and $Z$ inside this SESE
where at least one of $X$ and $Y$ is a definition.  But $X$ or $Y$
cannot be a definition of $v$ because they are inside the SESE of $Z$ which
was chosen to contain no definitions of $v$.
\end{description}

A symmetric argument holds for \sigfunction{s} for $v$, using
condition \ref{crit_sigplace} of the SSI form definition, and the fact
that there exists one exit edge from the SESE.
\end{proof}
}

\mycomment{
The above lemma justifies line~\ref{line:place_skip} of the algorithm
on page~\pageref{line:place_skip}, which skips over any SESE region
not containing a definition (use) of $v$ when placing
\phifunction{s} (\sigfunction{s}) for $v$.
}

\begin{lemma}\label{lem:ssi_place_dom}
If a definition (use) or a \phisigfunction[or]{} for a variable $v$ is
present at some node $D$ (\/$U$), then a \phifunction (\sigfunction) for
$v$ is needed at every node $N$:
\begin{enumerate}
\item of input (output) arity greater than 1,
\item reachable from $D$ (from which $U$ is reachable),
\item whose smallest enclosing SESE contains $D$ (\/$U$), and
\item which is not dominated by $D$ (not post-dominated by $U$).
\end{enumerate}
\end{lemma}
\begin{proof}
See Appendix~\ref{app:proofs}.
\end{proof}
\mycomment{
\begin{proof}
We will first prove that a node $N$ failing any one of the conditions does
not need a \phisigfunction[or].
\begin{itemize}
\item Conditions \ref{crit_phiplace} and \ref{crit_sigplace} of the
SSI form definition require node $N$ to be the first convergence
(divergence) of some paths $X \pathplus N$ and $Y \pathplus N$ ($N
\pathplus X$ and $N \pathplus Y$).  If the input arity is less than 2
or there is no path from a definition of $v$, than it fails the
$\phi$-placement criterion \ref{crit_phiplace}.  If the output arity
is less than 2 or there is no path to a use of $v$, then it fails the
$\sigma$-placement criterion \ref{crit_sigplace}.
\item If there exists a SESE containing $N$ that does not contain any
definition, \phisigfunction[or]{} $D$ for $v$, then $N$ does not require a
\phisigfunction[or]{} for $v$ by lemma~\ref{lem:sese_child}.
\item Let us suppose every $D_i$ containing a definition,
\phisigfunction[or]{} for $v$ dominates $N$.  If $N$ requires a
\phifunction for $v$,  there exist paths $D_1 \pathplus N$ and
$D_2 \pathplus N$ containing no nodes in common but $N$.  We use these
paths to construct simple paths $\code{START}\pathplus D_1 \pathplus N$ and
$\code{START}\pathplus D_2 \pathplus N$.  By the definition of a
dominator, every path from \code{START} to $N$ must contain every
$D_i$.  But $D_1 \pathplus N$ cannot contain $D_2$, and if
$\code{START} \pathplus D_1$ contains $D_2$, we can make a path
$\code{START} \pathplus D_2 \pathplus N$ which does not contain $D_1$
by using the $D_1$-free path $D_2 \pathplus N$.  The assumption leads
to a contradiction; thus, there must exist some $D_i$ which does not
dominate $N$ if $N$ is required to have a \phifunction for $v$.  The
symmetric argument holds for post-dominance and \sigfunction{s}.
\end{itemize}
This proves that the conditions are necessary.  It is obvious from an
examination of conditions \ref{crit_phiplace} and \ref{crit_sigplace}
of the SSI form definition and lemma \ref{lem:sese_child} that they are
sufficient.
\end{proof}
}

In practice, the conditions of Lemma \ref{lem:ssi_place_dom} are too expensive
to implement directly.  Instead, we use a conservative approximation
to SSI form, which allows us to place more \phisigfunction{s} than
minimal SSI requires 
\mycomment{(for example, a \phifunction for $v$ at the
circled node in Figure~\ref{fig:placeerror}), }
while satisfying the
conditions of the SSI form definition.  
Our algorithm also allows us to do pre-pruning of the SSI
form during placement.  The result is not pruned SSI, but contains a
tight superset of the \phisigfunction{s} that pruned form requires.
\mycomment{
\begin{myfigure}
\centering\renewcommand{\figscale}{0.30}\input{Figures/THmorephi}
\caption{An flowgraph where Algorithm~\ref{alg:SSIplace} places
\phifunction{s} conservatively.}\label{fig:placeerror}
\end{myfigure}
}

\begin{theorem}\label{thm:placeproof}
Algorithm~\ref{alg:SSIplace} places all the
\phisigfunction{s} required by the path-convergence criteria for
\phisigfunction{s}.
\end{theorem}
\begin{proof}
Lemma \ref{lem:sese_child} states that the child region exclusion of
Algorithm~\ref{alg:SSIplace} does not cause required \phisigfunction[or]{s} to
be omitted.  Property~\ref{pty:pruned_live} allows the omission of
\phisigfunction{s} for $v$ at nodes where $v$ is dead when creating
pruned form; \code{MaybeLive} may not return \code{false} for nodes
where $v$ is not dead, but may return \code{true} at nodes where $v$
is dead without harming the correctness of the \phisigfunction{}
placement.
\end{proof}
\dontfixme{It would be nice to be able to show a means of using the
algorithm and the conditions in \ref{lem:ssi_place_dom} to produce exactly
minimal form or exactly pruned form.  It doesn't hurt our time bounds
to do fixup later, though.}

\subsubsection{Computing liveness}
Incorporating liveness information into the creation of pruned SSI
form appears to lead to a chicken-and-egg problem: although the pruned
SSI framework allows highly efficient liveness analysis, obtaining the
liveness information from the original program can be problematic.
The fastest sparse algorithm has stated time bounds of $O(E+N^2)$
\cite{ferrante91:pruned}, which is likely to be more expensive than
the rest of the SSI form conversion.  Luckily, Kam and Ullman
\cite{kam76:dataflow}, in conjunction with an empirical study by Knuth
\cite{knuth74:fortran}, show that liveness analysis is highly likely
to be linear for reducible flow-graphs.  In our work this question is
avoided, as we obtain our liveness information directly from
properties of the Java bytecode files that are our input to the
compiler.  But in any case our algorithms allow conservative
approximation to liveness, so even in the case of non-reducible flow
graphs it should not be difficult to quickly generate a rough
approximation.

\subsubsection{Variable renaming}
\mycomment{
\begin{myfigure}[p]\small
% data types for SSI renaming algorithm.
Data type \code{Environment}:
\begin{tightdesc}
\item[create():~Environment]
:\par   make an environment with no mappings.
\item[put($\mathcal{E}$:~Environment, $v_1$:~variable, $v_2$:~variable)]
:\par   extend environment $\mathcal{E}$ with a mapping from $v_1$ to $v_2$.
\item[get($\mathcal{E}$:~Environment, $v$:~variable):~variable]
:\par   return the current mapping in $\mathcal{E}$ for $v$.
\item[beginScope($\mathcal{E}$:~Environment)]
:\par   save the current mapping of $\mathcal{E}$ for later restoration.
\item[endScope($\mathcal{E}$:~Environment)]
:\par   restore the mapping of $\mathcal{E}$ to that present at the
        last \code{beginScope} on $\mathcal{E}$.
\end{tightdesc}

\caption{Environment datatype for the SSI renaming algorithm.}
\label{fig:SSIrename_data}
\end{myfigure}
\begin{myalgorithm}\small
% algorithm for phi/sigma renaming.

Rename($G$:~CFG) =
\begin{myalgorithmic}
\STATE $\text{Init}(G)$
\LET $Q$ be an empty FIFO queue
      \EMPTYLINE
\FOREACH{edge $e$ leaving \code{START}}
 \STATE $\text{PushEdge}(e)$
\ENDFOR
\WHILE{$Q$ is not empty}
 \STATE pop edge $e$ from the start of $Q$
 \STATE $\text{Search}(e)$
 \ENDWHILE
\end{myalgorithmic}
~\\
Init($G$:~CFG) =
\begin{myalgorithmic}
\FOREACH{edge $e$ in $G$}
 \STATE $\text{Marked}[e] \gets \code{false}$
\ENDFOR
\FOREACH{variable $V$ in $G$}
 \STATE $C(V) \gets 0$
\ENDFOR
\end{myalgorithmic}
~\\
PushEdge($e$:~edge) =
\begin{myalgorithmic}
\IF{not $\text{Marked}[e]$}
 \STATE $\text{Marked}[e] \gets \code{true}$
 \STATE push edge $e$ onto the end of $Q$
\ENDIF
\end{myalgorithmic}
~\\
Get($V$:~variable):~variable =
\begin{myalgorithmic}
\STATE $i \gets C(V)$
\RETURN $V_i$
\end{myalgorithmic}
~\\
Inc($V$:~variable):~variable =
\begin{myalgorithmic}
\STATE $i \gets C(V) + 1$
\STATE $C(v) \gets i$
\RETURN $V_i$
\end{myalgorithmic}

\caption{SSI renaming algorithm.}\label{alg:SSIrename1}
\end{myalgorithm}
\begin{myalgorithm}\small
% SSI renaming algorithm, cont.
Search($\tuple{s,d}$:~edge) =
\begin{myalgorithmic}
\REQUIRE{$s$ to be a node containing \phisigfunction[or]{s}, or \code{START}}
\REQUIRE{$\text{Marked}[\tuple{s,d}] = \code{false}$}
\STATE $\text{Marked}[\tuple{s,d}] \gets \code{true}$
\STATE $\text{beginScope}(\mathcal{E})$
\IF{$s$ is a node containing \phifunction{s}}
 \FOREACH{\phifunction $P$ in $s$}
  \STATE replace the destination $V$ of $P$ by $\text{Inc}(\mathcal{E}, V)$
         \label{line:rendef1}
 \ENDFOR
\ELSIF{$s$ is a node containing \sigfunction{s}}
 \FOREACH{\sigfunction $S$ in $s$}
  \STATE $j \gets \text{WhichSucc}(\tuple{s,d})$
  \STATE replace the $j$-th destination $V$ of $S$ by $\text{Inc}(\mathcal{E}, V)$
         \label{line:rendef2}
 \ENDFOR
\ENDIF
\LOOP[now rename inside basic block] \label{line:searchloop_start}
 \IF{$d$ is a node containing \phifunction{s}}
  \FOREACH{\phifunction $P$ in $d$}
   \STATE $j\gets \text{WhichPred}(\tuple{s,d})$ \label{line:phisrc1}
   \STATE replace the $j$-th operand $V$ of $P$ by $\text{get}(\mathcal{E}, V)$
          \label{line:phisrc2}
  \ENDFOR
  \BREAK \COMMENT{end of basic block}
 \ELSIF{$d$ is a node containing \sigfunction{s}}
  \FOREACH{\sigfunction $S$ in $d$}
   \STATE replace the operand $V$ of $S$ by $\text{get}(\mathcal{E}, V)$
          \label{line:sigsrc}
  \ENDFOR
  \BREAK \COMMENT{end of basic block}
 \ENDIF
 \STATE\COMMENT{ordinary assignment, at most one successor}
        \label{line:rename_ordinary1}
 \FOREACH{variable $V$ in $RHS(d)$}
  \STATE replace $V$ by $\text{get}(\mathcal{E}, V)$ in $RHS(d)$
 \ENDFOR
 \FOREACH{variable $V$ in $LHS(d)$}
  \STATE replace $V$ by $\text{Inc}(\mathcal{E}, V)$ in $LHS(d)$
         \label{line:rendef3} \label{line:rename_ordinary2}
 \ENDFOR
 \IF{$d$ has no successor}
  \BREAK \COMMENT{end of basic block}
 \ENDIF
 \STATE $s \gets d$
 \STATE $d \gets \text{successor of }d$ \label{line:search_onesucc}
\ENDLOOP* \label{line:searchloop_end}
\FOREACH{successor $n$ of $d$}
 \IFNOT {$\text{Marked}[\tuple{d, n}]$}
  \STATE $\text{Search}(\tuple{d, n})$ \COMMENT{dfs recursion}
         \label{line:search2}
 \ENDIF
\ENDFOR
\STATE $\text{endScope}(\mathcal{E})$
\RETURN
\end{myalgorithmic}

\caption{SSI renaming algorithm, cont.}\label{alg:SSIrename2}
\end{myalgorithm}
}

We have shown that Algorithm~\ref{alg:SSIplace} places all the
required \phisigfunction{s} in the control-flow graph according to 
the path-convergence criteria for SSI form and the stated 
boundary conditions at \code{START} and \code{END}.
The next step is to rename variables to
be consistent with properties \ref{crit_phiname} and
\ref{crit_signame}. Algorithm~\ref{alg:SSIrename1} 
in Appendix~\ref{app:algorithms} performs
this variable renaming. 
Algorithm~\ref{alg:SSIrename1} starts on a
flow-graph with placed \phisigfunction{s}. 
When the algorithm finishes, the control flow-graph
will be in proper SSI form.  The SSI form is not necessarily
minimal. The next section will show how to post-process 
to create minimal or pruned SSI form.

\mycomment{
The algorithm requires
an \code{Environment} datatype which is defined in
Figure~\ref{fig:SSIrename_data}.  Using an imperative programming
style, it is possible to perform a sequence of any $N$ operations on
\code{Environment} as defined in the figure in $O(N)$ time; in a
functional programming style any $N$ operations can be completed in
$O(N \log N)$ time.\footnote{The curious reader is referred to section
5.1 of Appel \cite{appel:modern} for implementation details.}  As the
coarse structure of Algorithm~\ref{alg:SSIrename1} is a simple
depth-first search, it is easy to see that the \code{Search} procedure
can be invoked from line~\fullref{line:search1} and
line~\fullref{line:search2} a total of $O(E)$ times; likewise its
inner loop (lines~\ref{line:searchloop_start} to
\ref{line:searchloop_end}) can be executed a total of $E$ times across
all invocations of \code{Search}.  A total of $U_{SSA}+D_{SSA}$ calls
to the operations of the \code{Environment} datatype will be made
within all executions of \code{Search}.  For the imperative
implementation of \code{Environment} a total time bounds of
$O(E+U_{SSA}+D_{SSA})$ for the variable renaming algorithm is
obtained.

 to prove that these algorithms combined suffice to
convert a program into SSI form.  The SSI form is not necessarily
minimal, as we showed in section~\ref{sec:SSIplace}; the next
section will show how to post-process to create minimal or pruned SSI
form.
}


\mycomment{
\begin{lemma}\label{lem:path_construct}
The stack trace of calls to \code{Search} defines a unique path
through $G$ from \code{START}.
\end{lemma}
\begin{proof}
We will prove this lemma by construction.  For every consecutive pair
of calls to \code{Search} we construct a path $X\pathplus Y$ starting with the
edge $\tuple{X,N_0}$ which is the argument of the first call, and
ending with the edge $\tuple{N_n, Y}$ which is the argument of the
second call.  From line~\ref{line:search_onesucc} of the \code{Search}
procedure on page~\pageref{line:search_onesucc} we note that every
edge $\tuple{N_i, N_{i+1}}$ between the first and last has exactly one
successor.  Furthermore, the call to search on line~\ref{line:search2}
defines a path starting with the edge which our segment $X\pathplus Y$
ends with; therefore the paths can be combined.  By so doing from the
bottom of the call stack to the top we construct a unique path from
\code{START}.
\end{proof}

For brevity, we will hereafter refer to the canonical path constructed
in the manner of lemma~\ref{lem:path_construct} corresponding to the
stack of calls to \code{Search} when an edge $e$ is first
encountered as $CP(e)$.  Every edge in the CFG is encountered exactly
once by \code{Search}, so $CP(e)$ exists and is unique for every edge
$e$ in the CFG.

\begin{lemma}\label{lem:renamephi}
SSI form condition~\ref{crit_phiname} (\phifunction{} naming) holds
for variables renamed according to Algorithm~\ref{alg:SSIrename1}.
\end{lemma}
\begin{proof}
We restate SSI form condition~\ref{crit_phiname} for reference:
\begin{quote}
For every node $X$ containing a definition of a variable $V$ in
the new program and node $Y$ containing a use of that variable, there
exists at least one path $X \pathplus Y$ and no such path contains a
definition of $V$ other than at $X$.
\end{quote}
We consider the canonical path 
$CP(\tuple{Y',Y})=\code{START}\pathstar Y' \path Y$
for some use of a variable $v$ at $Y$, constructed according to
lemma~\ref{lem:path_construct} 
from a stack trace of calls to \code{Search}.
is encountered.  This path is unique, although more than one canonical
path may terminate at $Y$ at nodes with more than one predecessor.
These paths are distinguished by the incoming edge to
$Y$.\footnote{Note that the notation \tuple{N,N'} for denoting edges
does not always denote an edge unambigiously; imagine a conditional
branch where both the \code{true} and \code{false} case lead to the
same label.  In such cases an additional identifier is necessary to
distinguish the edges.  Alternatively, one may split such edges to
remove the ambiguity.  We treat edges as uniquely identifiable and
leave the implementation to the reader.}  We identify each operand
$v_i$ of a \phifunction{} with the appropriate incoming edge $e$ to
ensure that $CP(e)$ is well defined and unique in the context of a
use of $v_i$.

The canonical path $\code{START}\pathplus Y$ must contain $X$, a definition of
$v$, if $Y$ uses a variable defined in $X$, as \code{Search} renames
all definitions (in lines \ref{line:rendef1}, \ref{line:rendef2}, and
\ref{line:rendef3}) and destroys the name mapping in $\mathcal{E}$
just before it returns.  The call to \code{Search} which creates the
definition of $v$ must therefore always be on the stack, and thus in
the path $CP(\tuple{Y',Y})$, for any use to receive a the name $v$.
Note that this is
true for \phifunction{s} as well, which receive names when the
appropriate incoming edge $\tuple{Y',Y}$ is traversed, not necessarily
when the node $Y$ containing the \phifunction{} is first encountered.

We have proved that $\code{START}\pathplus X\pathplus Y$ exists; now
we must prove that no other path from $X$ to $Y$ contains a definition
of $v$.  Call this other definition $D$.  Obviously $D$ cannot be on
our canonical path $\code{START}\pathplus X\pathplus Y$, or
line~\ref{line:rendef3} would have caused $Y$ to use a different name.
But as we just stated, all variable name mappings done by $D$ will be
removed when the call to \code{Search} which touched $D$ is taken off
the call stack.  So $D$ must be on the call stack, and thus on the
canonical path; a contradiction. %
%
Since assuming the existence of some other path $X\pathplus Y$
containing a definition of $v$ leads to contradiction no other such
path may exist, completing the proof of the lemma.
\end{proof}

\begin{lemma}\label{lem:renamesig}
SSI form condition~\ref{crit_signame} (\sigfunction{} naming) holds for
variables renamed according to Algorithm~\ref{alg:SSIrename1}.
\end{lemma}
\begin{proof}
We restate SSI form condition~\ref{crit_signame} for reference:
\begin{quote}
For every pair of nodes $X$ and $Y$ containing uses of a
variable $V$ defined at node $Z$ in the new program, either every path
$Z \pathplus X$ must contain $Y$ or every path $Z \pathplus Y$ must
contain $X$.
\end{quote}
Let us assume there are paths $Z\pathplus X$ and $Z\pathplus Y$
violating this condition; that is, let us chose nodes $X$ and $Y$
which use $V$ and $Z$ defining $V$ such that there exists a path $P_1$
from $Z$ to $X$ not containing $Y$ and a path $P_2$ from $Z$ to $Y$ not
containing $X$.  By the argument of the previous lemma, there exists
a canonical path $P_3=CP(e)$ from \code{START} to $X$ through $Z$
corresponding to a stack
trace of \code{Search}; note that $P_3$ need not contain $P_1$.
There are two cases:
\begin{description}
\item[Case I:] $P_3$ does not contains $Y$.  Then there is some last
node $N$ present on both $P_2: Z\pathstar N\pathplus Y$ and
$P_3: \code{START}\pathplus Z\pathstar N\pathplus X$.  By SSI
condition~\ref{crit_sigplace} this node $N$ requires a \sigfunction{}
for $V$.  If $N\not=Z$ then line~\ref{line:rendef1} of
Algorithm~\ref{alg:SSIrename1} would rename $V$ along $P_3$
and $X$ would not use the same variable $Z$ defined; if
$N=Z$, then line~\ref{line:rendef2} would have ensured that $X$
and $Y$ used different names.  Either case contradicts our choices of
$X$, $Y$, and $Z$.
\item[Case II:] $P_3$ does contain $Y$.  Then consider the path
$\code{START}\pathplus Z\pathplus Y$ along $P_3$, which does not
contain $X$.  The argument of case I applies with $X$ and $Y$ reversed.
\end{description}
Any assumed violation of condition~\ref{crit_signame} leads to
contradiction, proving the lemma.
\end{proof}

Every path $CP(e)$ corresponds to a execution state in a call to
\code{Search} at the point where $e$ is first encountered.  The value
of the environment mapping $\mathcal{E}$ at this point in the
execution of Algorithm~\ref{alg:SSIrename1} we will denote as
$\mathcal{E}^e$.  For a node $N$ having a single predecessor $N_p$ and
single successor $N_s$, we will denote
$\mathcal{E}^{\tuple{N_p,N}}$ as $\mathcal{E}_{\text{before}}^N$ and 
$\mathcal{E}^{\tuple{N,N_s}}$ as $\mathcal{E}_{\text{after}}^N$.
It is obvious that 
$\mathcal{E}_{\text{after}}^{N_p} = \mathcal{E}_{\text{before}}^{N  }$ and
$\mathcal{E}_{\text{after}}^{N  } = \mathcal{E}_{\text{before}}^{N_s}$
when $N_p$ and $N_s$, respectively, are also single-predecessor
single-successor nodes.

\begin{lemma}\label{lem:correctness}
SSI form condition~\ref{crit_correct} (correctness) holds for
variables renamed according to Algorithm~\ref{alg:SSIrename1}.  That
is, along any possible control-flow path in a program being executed a
use of a variable $V_i$ in the new program will always have the same
value as a use of the corresponding variable $V$ in the original
program.
\end{lemma}
\begin{proof}
We will use induction along the path $N_0\path N_1\path\ldots\path N_n$.
We consider $e_k=\tuple{N_{k},N_{k+1}}$, the $(k+1)$th edge in the path,
and assume that, for all $j<k$, each variable $V$ in the original
program agrees with the value of $\mathcal{E}^{e_j}[V]=V_i$ in the new
program.  We show that $\mathcal{E}^{e_k}[V]$ agrees with $V$ at edge
$e_k$ in the path.
\begin{description}
\item[Case I:] $k=0$. The base case is trivial: the \code{START} node
($N_0$) contains no statements, and along each edge $e$ leaving start
$\mathcal{E}^e[V]=V_0$.  By definition $V_0$ agrees with $V$ at the
entry to the procedure.
\item[Case II:] $k>0$ and $N_k$ has exactly one predecessor and one successor.
If $N_k$ is single-entry single-exit, then it is not a \phisigfunction[or].
As an ordinary assignment, it will be handled by
lines~\ref{line:rename_ordinary1} to \ref{line:rename_ordinary2} of
Algorithm~\vref{alg:SSIrename2}.  By the induction hypothesis (which
tells us that the uses at $N_k$ correspond to the same values as the
uses in the original program) and the semantics of
assignment, the mapping $\mathcal{E}_{\text{after}}^{N_k}$ is easily
verified to be valid when $\mathcal{E}_{\text{before}}^{N_k}$ is
valid.  Thus the value of every original variable $V$ corresponds to
the value of the new variable 
$\mathcal{E}_{\text{after}}^{N_k}[V]=\mathcal{E}^{e_k}[V]$ on $e_k$.
\item[Case III:] $k>0$ and $N_k$ has multiple predecessors and one
successor.  In this case $N_k$ may have multiple \phifunction{s} in
the new program, and \mycomment{by the definition in section~\ref{sec:defs}} $N_k$
has no statements in the original program.  Thus the value of any
variable $V$ in the original program along edge $e_k$ is identical to
its value along edge $e_{k-1}$.  We need only show that the value of
the variable $\mathcal{E}^{e_{k-1}}[V]$ is the same as the value of
the variable $\mathcal{E}^{e_k}[V]$ in the new program.  For any
variable $V$ not mentioned in a \phifunction{} at $N_k$ this is
obvious.  Each variable defined in a \phifunction{} will get the value
of the operand corresponding to the incoming control-flow path edge.
The relevant lines in Algorithm~\ref{alg:SSIrename2} start with
\ref{line:phisrc1} and \ref{line:phisrc2}, where we see that the
operand corresponding to edge $e_{k-1}$ of a \phifunction{} for $V$
correctly gets $\mathcal{E}^{e_{k-1}}[V]$.  At
line~\ref{line:rendef1}, we see that the destination of the
\phifunction{} is correctly $\mathcal{E}^{e_k}[V]$.  Thus the value of
every original variable $V$ correctly correponds to
$\mathcal{E}^{e_k}[V]$ by the induction hyptothesis and the semantics
of the \phifunction{s}.
\item[Case IV:] $k>0$ and $N_k$ has one predecessor and multiple
successors.  Here $N_k$ may have multiple \sigfunction{s} in the new
program, and is empty in the original program.  The argument goes as
for the previous case.  It is obvious that variables not mentioned in
the \sigfunction{s} correspond at $e_k$ if they did at $e_{k-1}$.  For
variables mentioned in \sigfunction{s}, line~\ref{line:sigsrc} shows
that operands correctly get $\mathcal{E}^{e_{k-1}}[V]$ and
line~\ref{line:rendef2} shows that the destination corresponding to
$e_k$ correctly gets $\mathcal{E}^{e_k}[V]$.  Therefore the values of
original variables $V$ correspond to the value of
$\mathcal{E}^{e_k}[V]$ by the induction hypothesis and the semantics
of the \sigfunction{s}.
\comment{
\item[Case V:] $N_k$ has multiple predecessors and multiple
successors.  Forbidden by the CFG definition in section~\ref{sec:defs}.
\end{description}
Therefore, on every edge of the chosen path, the values of the
original variables correspond to the values of the renamed SSI form
variables. The value correspondence at the path endpoint (a use of
some variable $V$) follows.
\end{proof}
}
}

\begin{theorem}\label{thm:renameproof}
Algorithm~\ref{alg:SSIrename1} renames variables such that SSI form
properties \ref{crit_phiname}, \ref{crit_signame}, and
\ref{crit_correct} hold.
\end{theorem}
\begin{proof}
Direct from lemmas~\ref{lem:renamephi}, \ref{lem:renamesig}, and
\ref{lem:correctness}.
\end{proof}

\begin{theorem}
Algorithms~\ref{alg:SSIplace} and \ref{alg:SSIrename1} correctly
transform a program into SSI form.
\end{theorem}
\begin{proof}
Theorem~\ref{thm:placeproof} proves that \phisigfunction{s} are placed
correctly to satisfy the path-convergence criteria of the SSI form
definition, and theorem~\ref{thm:renameproof} proves that variables
are renamed correctly to satisfy properties~\ref{crit_phiname},
\ref{crit_signame} and~\ref{crit_correct}.
\end{proof}

\subsubsection{Pruning SSI form}\label{sec:unusedcode}
The SSI algorithm can be run using any conservative approximation to
the liveness information
(including the function $\code{MaybeLive}(v, n)=\code{true}$) if
unused code elimination%
\footnote{We follow \cite{wegman91:scc} in distinguishing
\emph{unreachable code elimination}, which removes code that can never
be executed, from \emph{unused code elimination}, which deletes
sections of code whose results are never used.  Both are often called
``dead code elimination'' in the literature.} is performed to remove
extra \phisigfunction{s} added and create pruned SSI.
Figure~\ref{fig:deaddata} and Algorithm~%
\ref{alg:deadalg} present an algorithm to identify unused code in
$O(N V_{SSI})$ time, after which a simple $O(N)$ pass suffices to remove it.
The complexity analysis is simple: nodes and variables are visited at
most once, raising their value in the analysis lattive from
\emph{unused} to \emph{used}.  Nodes marked \emph{used} are never
visted.  So \code{MarkNodeUseful} is invoked at most $N$ times, and
\code{MarkVarUseful} is invoked at most $V_{SSI}$ times.  The calls to
\code{MarkNodeUseful} may examine at most every variable use in the
program in lines~\ref{line:deadnode1}-\ref{line:deadnode2}, taking
$O(U_{SSI})$ time at worst. Each call
to \code{MarkVarUseful} examines at most one node (the single
definition node for the variable, if it exists) and in constant time
pushes at most one node on to the worklist for a total of $O(V_{SSI})$ time.
So the total run time of \code{FindUseful} is
$O(U_{SSI}+V_{SSI})=O(U_{SSI})$.
%\footnote{If the number of instruction
%operands and \phisigfunction{} arities are limited by a
%constant, we get a time bound of $O(N)$.}

\mycomment{
\begin{myfigure}\small
% data types for cycle-equivalency algorithm.
Operations on nodes:
\begin{tightdesc}
\item[NodeUseful(n:node):~boolean]
:	Whether the results of this node are ever used
\item[Uses(n:node):~set of variables]
:	Variables for which this node contains a use
\end{tightdesc}

Operations on variables:
\begin{tightdesc}
\item[VarUseful(v:variable):~boolean]
:	Whether there is some \textbf{n} for which 
        \textbf{Uses(n)} contains \textbf{v} and
        \textbf{NodeUseful(n)} is \texttt{true}
%:	$\exists n\in\domain{Nodes}\;.\;v\in\mbox{Uses}(n)\wedge\mbox{NodeUseful}(n)$
%:	Whether the value of this variable is ever used
\item[Definitions(v:variable):~set of nodes]
:	Nodes which contain a definition for~\textbf{v}
\end{tightdesc}

\caption{Datatypes and operations used in unused code elimination.}
\label{fig:deaddata}
\end{myfigure}

\begin{myalgorithm}\small\linespread{0.75}
% fast algorithm for dead code elimination (unused code elimination)
\begin{verbatim}
Procedure FindUseful (G: CFG) {
  let W be an empty work list;
  for each variable v, do
    VarUseful(v) := false;
  for each node n in G, in any order, do {
    NodeUseful(n) := false;
    if n is a CALL, RETURN, or other special node, then
      add n to W;
  }

  while W is not empty, do {
    let n be any element from W;
    remove n from W;
    MarkNodeUseful(n, W);
  }
}
Procedure MarkNodeUseful(n: node, W: WorkList) {
  NodeUseful(n) := true;
  /* everything used by a useful node is useful */
  for each v in Uses(n), do
    if not VarUseful(v), then
      MarkVarUseful(v, W);
}
Procedure MarkVarUseful(v: variable, W: WorkList) {
  VarUseful(v) := true;
  /* The definition of a useful variable is useful */
  for each n in Definitions(v), then
    /* size(Definitions(v)) <= 1 in SSI form */
    if not NodeUseful(n), then
      add n to W;
}
\end{verbatim}
\caption{Identifying unused code using SSI form.}
\label{alg:deadalg}
\end{myalgorithm}
}

\subsubsection{Discussion}
Note that our algorithm for placing \phisigfunction{s} in SSI form is
\emph{pessimistic}; that is, we at first assume every node in the
control-flow graph with input arity larger than one requires a
\phifunction{} for every variable and every node with out-arity larger
than one requires a \sigfunction{} for every variable, and then use
the PST, liveness information, and unused code elimination to
determine safe places to
\emph{omit} \phisigfunction[or]{s}.  Most SSA construction
algorithms, by contrast, are \emph{optimistic}; they assume no
\phisigfunction[or]{s} are needed and attempt to determine where
they are provably necessary.  
In our experience, optimistic algorithms tend to have poor
time bounds because, in the worst case, they 
may need to perform multiple
passes over the graph as they propagate \phisigfunction[or]{s}.
In such cases, a pessimistic algorithm assumes the correct answer at the 
start, fails to show that any \phisigfunction[or]{s} can be removed, and
terminates in one pass. See Appendix~\ref{app:optimistic} for more
information.

\mycomment{
In our experience, optimistic algorithms tend to
have poor time bounds because of the possibility of input graphs like
the one illustrated in Figure~\vref{fig:evil}.
Proving that all but two nodes require
\phisigfunction[and/or]{s} for the variable $a$ in this example seems to
inherently require $O(N)$ passes over the graph; each pass can prove
that \phisigfunction[or]{s} are required for only those nodes adjacent to
nodes tagged in the previous pass.  Starting with the circled node, the
\phisigfunction{s} spread one node left on each pass. On the other hand,
a pessimistic algorithm assumes the correct answer at the start, fails
to show that any \phisigfunction[or]{s} can be removed, and
terminates in one pass.\dontfixme{Are we \emph{sure} similar worst cases
don't exist for the pessimistic algorithm?}

\begin{myfigure}[t]
\centering\renewcommand{\figscale}{0.25}\input{Figures/evil}
\caption{A worst-case CFG for ``optimistic'' algorithms.}
\label{fig:evil}
\end{myfigure}
}

\subsection{Time and space complexity of SSI form}\label{sec:ssi_complexity}
%\begin{myfigure}%[t]
%\input{Figures/phisig}
%\caption{Number of \phisigfunction{s} added versus procedure length.}
%\label{fig:phisigdata}
%\end{myfigure}
\begin{myfigure}%[t]
\centering\renewcommand{\figscale}{0.7}\input{Figures/THussi}
\caption{Number of uses in SSI form as a function of
procedure~length.}
\label{fig:ussidata}
\end{myfigure}
\begin{myfigure}%[t]
\centering\renewcommand{\figscale}{0.7}\input{Figures/THv0}
\caption{Number of original variables as a function of
procedure~length.}
\label{fig:v0data}
\end{myfigure}
Discussions of time and space complexity for sparse evaluation
frameworks in the literature are often misleadingly called ``linear''
regardless of what the $O$-notation runtime bounds are.  A canonical
example is \cite{sreedhar95:lintime}, which states that
for SSA form, ``the number of $\phi$-nodes needed remains linear.''
Typically Cytron \cite{cytron91:ssa} is cited; however, that reference
actually reads:
\begin{quote}
For the programs we tested, the plot in [Figure 21 of Cytron's paper]
shows that the number of \phifunction{s} is also linear in the size of
the original program.
\end{quote}
It is important to note that Cytron's claim is based not on
algorithmic worst-bounds complexity, but on empirical evidence.  This
reasoning is not unjustified; Knuth \cite{knuth74:fortran} showed in
1974 that ``human-generated'' programs almost without exception show
properties favorable to analysis; in particular shallow maximum loop
nesting depth.  Wegman and Zadeck \cite{wegman91:scc} clearly make
this distinction by noting that:
\begin{quote}
In theory the size [of the SSA form representation] can be $O(EV)$,
but empirical evidence indicates that the work required to compute the
SSA graph is linear in the program size.
\end{quote}
Our worst-case space complexity bounds for SSI form are identical to
SSA form --- $O(EV)$ --- but in this section we will endeavour to show
that typical complexities are likewise ``linear in the program size.''

The total runtime for SSI placement and subsequent pruning, including
the time to construct the PST, is $O(E + N V_0 + U_{SSI})$.  For most
programs $E$ will be a small constant factor multiple of $N$; as
Wegman and Zadeck \cite{wegman91:scc} note, most control flow graph
nodes will have at most two successors.  For those graphs where $E$ is
not $O(N)$, it can be argued that $E$ is the more relevant measure of
program complexity.
\mycomment{
\footnote{We will not follow Cytron \cite{cytron91:ssa} in
defining a new variable $R$ to denote $\max(N,E,\ldots)$ to avoid
following him in declaring worst-case complexity $O(R^3)$ and leaving
it to the reader to puzzle out whether $O(N^6)$ (!) is really being implied.}
}

Thus the ``linearity'' of our SSI construction algorithm rests on the
quantities $N V_0$ and $U_{SSI}$.  Figures~\ref{fig:ussidata} and
\ref{fig:v0data} present empirical data for $V_0$ and $U_{SSI}$ on a
sample of 1,048 Java methods.  The methods varied in length from 4
to 6,642 statements and were taken from the dynamic call-graph of the
FLEX compiler itself, which includes large portions of the standard
Java class libraries.  Figure~\ref{fig:ussidata} shows convincingly
that $U_{SSI}$ grows as $N$ for large procedures, and
Figure~\ref{fig:v0data} supports an argument that $V_0$ grows very
slowly and that the quantity $N V_0$ would tend to grow as $N^{1.3}$.
This would argue for a near-linear practical run-time.

In contrast, Cytron's original algorithm for SSA form had theoretical
complexity $O(E + V_{SSA} |\text{DF}| + N V_{SSA})$.  Cytron does not
present empirical data for $V_{SSA}$, but one can infer from the data
he presents for ``number of introduced \phifunction{s}'' that
$V_{SSA}$ behaves similarly to $V_{SSI}$ --- that is, it grows as $N$,
not as $V_0$.  It is frequently pointed out\footnote{See Dhamdhere
\cite{dhamdhere92:large} for example.} that the $|\text{DF}|$
term, the size of the dominance frontier, can be $O(N^2)$ for common
programming constructs (\code{repeat-until} loops), which indicates that
the $V_{SSA} |\text{DF}|$ term in Cytron's algorithm will be $O(N^2)$
at best and at times as bad as $O(N^3)$.

Note that the space complexity of SSI form, which may be $O(EV)$ in
the worst case (\phisigfunction{s} for every variable inserted at
every node) is certainly not greater than $U_{SSI}$, and thus
Figure~\ref{fig:ussidata} shows linear practical space use.

\section{Uses and applications of SSI}
The principle benefits of using SSI form are the ability to do
predicated and backward dataflow analyses efficiently.
\newterm{Predicated analysis} means that we can use information
extracted from branch conditions and control flow.  The
\sigfunction{s} in SSI form provide an variable naming that
allows us to sparsely associate the predication information with
variable names at control flow splits.  The \sigfunction{s} also
provide a reverse symmetry to SSI form that allow efficient backward
dataflow analyses like \newterm{liveness} and
\newterm{anticipatability}.

In this section, we will briefly sketch how SSI form can be applied to
backwards dataflow analyses, including anticipatability, an important
component of partial redundancy elimination.  We will then describe in
detail our Sparse Predicated Typed Constant propagation algorithm,
which shows how the predication information of SSI form may be used to
advantage in practical applications, including the removal of array
bounds and null-pointer checks.  Lastly, we will describe an extension
to SPTC that allows \newterm{bitwidth analysis}, and the possible uses
of this information.

\subsection{Backward Dataflow Analysis}\label{sec:bidirectional}
\newterm{Backward dataflow analyses} are those in which information is
propagated in the direction opposite that of program execution
\cite{offner95}.  There is general agreement
\cite{johnson93:dfg,ferrante91:pruned,weise94:vdg}
that SSA form is unable to directly handle backwards dataflow
analyses; \newterm{liveness} is often cited as a canonical example.

However, SSI form allows the sparse computation of such backwards
properties.  Liveness, for example, comes ``for free'' from pruned SSI
form: every variable is live in the region between its use and
sole definition.  Every
non-\phifunction{} use of a variable is dominated by the definition;
Cytron \cite{cytron91:ssa} has shown that \phifunction{s} will always be
found on the dominance frontier.  Thus the live region between
definition and use can be enumerated with a simple depth-first search,
taking advantage of the topological sorting by dominance that DFS
provides \cite{offner95}.  Because of \phifunction{} uses, the DFS
will have to look one node past its spanning-tree leaves to see the
\phifunction{s} on the dominance frontier; this does not change the
algorithmic complexity.

Computation of other dataflow properties will use this same
enumeration routine to propagate values computed on the sparse SSI
graph to the intermediate nodes on the control-flow graph.  Formally,
we can say that the dataflow property for variable $v$ at node $N$ is
dependent only on the properties at nodes $D$ and $U$, defining and
using $v$, for which there is a path $D\pathplus U$ containing $N$.
There is a ``default'' property which holds for nodes on no such path
from a definition to use; for liveness the default property is ``not
live.''  The remainder of this section will concentrate on the
dataflow properties at use and definition points.

A slightly more complicated backward dataflow property is
\newterm{very busy expressions}; this analysis is somewhat obsolete as
it serves to save code space, not time.  This in turn is related to
partial and total \newterm{anticipatability}.

\begin{definition}
An expression $e$ is \newterm{very busy} at a point $P$ of the program iff it
is always subsequently used before it is killed \cite{offner95}.
\end{definition}
\begin{definition}
An expression $e$ is \newterm{totally (partially) anticipatable} at a
point $P$ if, on every (some) path in the CFG from $P$ to \code{END},
there is a computation of $e$ before an assignment to any of the
variables in $e$ \cite{johnson93:dfg}.
\end{definition}

Johnson and Pingali \cite{johnson93:dfg} show how to reduce these
properties of expressions to properties on variables.  We will
therefore consider properties $\text{BSY}(v,N)$, $\text{ANT}(v,N)$,
and $\text{PAN}(v,N)$ denoting very busy, totally anticipatable, and
partially anticipatable variables $v$ at some program point $N$.
To compute BSY, we start with pruned SSI form.  Any variable defined
in a \phisigfunction[or]{} is used at some point, by definition.
So for statements at a point $P$ we have the rules:
\begin{displaymath}
\begin{array}{ll}
v = \ldots & \text{BSY}_{\text{in}}(v,P)=\code{false} \\
\ldots = v & \text{BSY}_{\text{in}}(v,P)=\code{true} \\
x=\phi(y_0,\ldots,y_n) &
       \text{BSY}_{\text{in}}(y_i,P)=\text{BSY}_{\text{out}}(x,P) \\
\tuple{x_0,\ldots,x_n}=\sigma(y) &
       \text{BSY}_{\text{in}}(y,P)=\bigwedge_{i=0}^n \text{BSY}_{\text{out}}(x_i,P) \\
\end{array}
\end{displaymath}

Total anticipatability, in the single variable case, is identical to
BSY.  Partial anticipatability for a variable $v$ at point $P$ follows
the rules:
\begin{displaymath}
\begin{array}{ll}
v = \ldots & \text{PAN}_{\text{in}}(v,P)=\code{false} \\
\ldots = v & \text{PAN}_{\text{in}}(v,P)=\code{true} \\
x=\phi(y_0,\ldots,y_n) &
       \text{PAN}_{\text{in}}(y_i,P)=\text{PAN}_{\text{out}}(x,P) \\
\tuple{x_0,\ldots,x_n}=\sigma(y) &
       \text{PAN}_{\text{in}}(y,P)=\bigvee_{i=0}^n \text{PAN}_{\text{out}}(x_i,P) \\
\end{array}
\end{displaymath}

The present section is concerned more with feasibility than the
mechanics of implementation; we refer the interested reader to
\cite{offner95} and \cite{johnson93:dfg} for details on how to turn
the efficient computation of BSY, PAN and ANT into practical
code-hoisting and partial-redundancy elimination routines, respectively.

We note in passing that the sophisticated strength-reduction and code-motion
techniques of SSAPRE \cite{kennedy98:strength} are applicable to an SSI-based
representation, as well, and may benefit from the predication
information available in SSI.  The remainder of this section will
focus on practical implementations of predicated analyses using SSI form.

\subsection{Sparse Predicated Typed Constant Propagation}
Sparse Predicated Typed Constant (SPTC) Propagation is a powerful
analysis tool which
derives its efficiency from SSI form.  It is built on Wegman and
Zadeck's Sparse Conditional Constant (SCC) algorithm
\cite{wegman91:scc} and removes unnecessary array-bounds and
null-pointer checks, computes variable types, and performs
floating-point- and string-constant-propagation in addition to the
integer constant propagation of standard SCC.

We will describe this algorithm incrementally, beginning with the
standard SCC constant-propagation algorithm.
Wegman and Zadeck's algorithm operates on a program in SSA form; we will
call this SCC/SSA to differentiate it from SCC/SSI, which uses the SSI
form.  Section
\vref{sec:bitwidth} will discuss an extension to SPTC which does
\newterm{bit-width analysis}.

\subsubsection{Wegman and Zadeck's SCC/SSA algorithm}
\begin{myfigure}
\centering\renewcommand{\figscale}{0.5}\input{Figures/THlat1}
\caption[Value and executability lattices for SCC.]
{Three-level value lattice and two-level executability lattice for SCC.}
\label{fig:scclat1}
\end{myfigure}
\begin{mytable}\centering
$\begin{array}{|l|cccc|} \hline
\meet & \bot & c & d (\not= c) & \top \\ \hline
\bot  & \bot & c & d  & \top \\
c     &   c  & c &\top & \top \\
\top  & \top &\top&\top& \top \\ \hline
\end{array}%
\quad\quad%
\begin{array}{|l|ccc|}\hline
\oplus& \bot &   d  & \top \\ \hline
\bot  & \bot &   d  & \top \\
c     &   c  &c\oplus d& \top \\
\top  & \top & \top & \top \\ \hline
\end{array}$
\caption{Meet and binary operation rules on the SCC value lattice.}
\label{tab:sccmeet1}
\end{mytable}

\mycomment{
\begin{myalgorithm}\small
Init($G$:CFG) =
\begin{algorithmic}
\STATE $E_e \gets \emptyset$
\STATE $E_n \gets \emptyset$
\FOREACH {variable $v$ in $G$}
 \IF{some node $n$ defines $v$}
  \STATE $V[v] \gets \bot$
 \ELSE
  \STATE $V[v] \gets \top$
 \ENDIF
\ENDFOR
\end{algorithmic}
~\\
Analyze($G$:CFG) =
\begin{algorithmic}
\LET $r$ be the start node of graph $G$
\STATE $E_n \gets E_n \cup \{ r \}$
\STATE $W_n \gets \{ r \}$
\STATE $W_v \gets \emptyset$
          \EMPTYLINE
\REPEAT
 \IF{$W_n$ is not empty}
  \STATE remove some node $n$ from $W_n$
  \IF{$n$ has only one outgoing edge $e$ leading to some node $n'$}
   \STATE RaiseE($e$)
  \ENDIF
  \STATE Visit($n$)
 \ENDIF
 \IF{$W_v$ is not empty}
  \STATE remove some variable $v$ from $W_v$
  \FOREACH{node $n$ containing a use of $v$}
   \STATE Visit($n$)
  \ENDFOR
 \ENDIF
\UNTIL{both $W_v$ and $W_n$ are empty}
\end{algorithmic}

\caption{SCC algorithm for SSA form.}\label{alg:scc}
\end{myalgorithm}
\begin{myalgorithm}\small
RaiseE($e$:edge) =
\begin{myalgorithmic}
\STATE\COMMENT{When called, $e \notin E_e$}
\STATE $E_e \gets E_e \cup \{ e \}$
\LET $n$ be the destination of edge $e$
\IF{$n \notin E_n$}
\STATE $E_n \gets E_n \cup \{ n \}$
\STATE $W_n \gets W_n \cup \{ n \}$
\ENDIF
\end{myalgorithmic}
~\\
RaiseV($v$:variable, $L$:lattice value) =
\begin{myalgorithmic}
\IF{$V[v] \latlt L$}
\STATE $V[v] \gets L$
\STATE $W_v \gets W_v \cup \{ v \}$
\ENDIF
\end{myalgorithmic}
~\\
Visit($n$:node) =
\begin{myalgorithmic}
\FOREACH{assignment ``$v \gets x \oplus y$'' in $n$}
 \STATE RaiseV($v$, $V[x] \oplus V[y]$)
 \COMMENT{binop rule: see table~\ref{tab:sccmeet1}}
\ENDFOR
               \EMPTYLINE
\FOREACH{assignment ``$v \gets \text{MEM}(\ldots)$'' or 
                    ``$v \gets \text{CALL}(\ldots)$'' in $n$}
 \STATE RaiseV($v$, $\top$)
\ENDFOR
               \EMPTYLINE
\FOREACH{assignment ``$v \gets \phi(x_1,\ldots,x_n)$'' in $n$}
 \FOREACH{variable $x_i$ corresponding to predecessor edge $e_i$ of $n$}
  \IF{$e_i \in E_e$}
   \STATE RaiseV($v$, $V[v] \meet V[x_i]$)
   \COMMENT{meet rule: see table~\ref{tab:sccmeet1}}
  \ENDIF
 \ENDFOR
\ENDFOR
               \EMPTYLINE
\FOREACH{branch ``if $v$ goto $e_1$ else $e_2$'' in $n$}
 \STATE $L \gets V[v]$
 \IF{$L=\top$ or $L=c$ where $c$ signifies ``true'' and $e_1 \notin E_e$}
  \STATE RaiseE($e_1$)
 \ENDIF
 \IF{$L=\top$ or $L=c$ where $c$ signifies ``false'' and $e_2 \notin E_e$}
  \STATE RaiseE($e_2$)
 \ENDIF
\ENDFOR
\end{myalgorithmic}
\caption{SCC algorithm for SSA form, cont.}\label{alg:scc2}
\end{myalgorithm}
}

The SCC algorithm works on a simple three-level value lattice
associated with variable definition points and a two-level
executability lattice associated with flow-graph edges.  These
lattices are shown in Figure~\vref{fig:scclat1}.  
The SCC algorithm itself,
which runs in $O(E+U_{SSA})$ time, is presented in Figures~\ref{alg:scc}
and \ref{alg:scc2} from Appendix~\ref{app:algorithms}.

\mycomment{
Associating a lattice
value with a definition point is a conservative statement that, for
all possible program paths, the value of that variable has a certain
property.  The value lattice is, formally, $\domain{Int}_\bot^\top$;
the lattice value $\bot$ signifies that no information
about the value is known, the lattice value $\top$ indicates that it
is possible that the variable has more than one dynamic value, and the
other lattice entries (corresponding to integer constants and occuping
a flat space between $\top$ and $\bot$) indicate that the variable can
be proven to have a single constant value in all runs of the program.%
\footnote{Note that we follow the $\top$ and $\bot$ conventions used
in semantics and abstract interpretation; authors in dataflow analysis
(including Wegman and Zadeck in their SCC paper \cite{wegman91:scc})
often use contrary definitions, letting $\top$ mean undefined and
$\bot$ indicate overdefinition.  As section \ref{sec:semantics} will
discuss the semantics of \ssiplus at length, we thought it best to
adhere to one set of definitions consistently, instead of switching
mid-paper.}
Similarly, the executability lattice indicates whether it is possible
that the control flow edge is traversed in some execution of the
program (marked ``executable''), or if it can be proven that the edge
is never traversed in any valid program path (marked ``not
executable'').  The algorithm works with SSA form, and is presented
as Algorithm~\ref{alg:scc}.  Binary operations on lattice values and
combination at $\phi$-nodes follow the rules in
Table~\ref{tab:sccmeet1}; notice that the meet operation ($\meet$) is
simply the least upper bound on the lattice.
The time complexity of SCC/SSA can be found
easily: the procedure \code{RaiseE} puts each node on the $W_n$
worklist at most once, and \code{RaiseV} puts a variable on the $W_v$
worklist at most $D-1$ times, where $D$ is the maximum lattice depth.
The \code{Visit} procedure can thus be invoked a maximum of $N$ times
by line~\ref{line:visitWn} of the \code{Analyze} procedure of
Algorithm~\ref{alg:scc}, and a maximum of $U_{SSA}(D-1)$ times by
line~\ref{line:visitWv}, where $U_{SSA}$ is the number of variable
\newterm{uses} in the SSA representation of the program.  The lattice
depth $D$ is the constant 3 in this version of the algorithm, so it
drops out of the expression.  The \code{RaiseE} procedure itself is
called at most $E$ times.  The time complexity is thus
$O(E+N+U_{SSA}(D-1))$ which simplifies to $O(E+U_{SSA})$.
}

\subsubsection{SCC/SSI: predication using \sigfunction{s}.}\label{sec:sccssi}
\mycomment{
\begin{myfigure}%
\begin{samplecode}[2]%
foo = f();        & \subvar{foo}{0} = f();\\
if (foo == 1)     & if (\subvar{foo}{0} == 1) \\
                  & $\tuple{\subvar{foo}{1},\subvar{foo}{2}}$ =
                    $  \sigma(\subvar{foo}{0})$ \\
\>bar = foo + 1;  & \>\subvar{bar}{0} = \subvar{foo}{2} + 1;\\
else              & else \\
\>bar = 2;        & \>\subvar{bar}{1} = 2;\\
                  & \subvar{bar}{2} =
                    $  \phi(\subvar{bar}{0},\subvar{bar}{1})$%\\
\end{samplecode}%
\caption{A simple constant-propagation example.}
\label{fig:ssa_vs_ssi}
\end{myfigure}
}

\begin{myalgorithm}\small
Visit($n$:node) =
\begin{myalgorithmic}
\STATE \COMMENT{Assignment rules as on page \pageref{alg:scc2}}
               \EMPTYLINE
\FOREACH{branch ``if $x=y$ goto $e_1$ else $e_2$'' in $n$}
 \IF{$L[x]=\top$ or $L[y]=\top$}
  \STATE RaiseE($e_1$)
  \STATE RaiseE($e_2$)
 \ELSIF{$L[x]=c$ and $L[y]=d$}
  \IF{$c=d$}
   \STATE RaiseE($e_1$)
  \ELSE
   \STATE RaiseE($e_2$)
  \ENDIF
 \ENDIF
\FOREACH{assignment ``$\tuple{v_1,v_2}\gets\sigma(v_0)$'' associated
         with this branch}
 \IF{edge $e_1 \in E_e$ and variable $v_0$ is the $x$ or $y$ in the test}
  \STATE RaiseV($v_1$, min($L[x]$, $L[y]$))
 \ELSIF{edge $e_1 \in E_e$}
  \STATE RaiseV($v_1$, $L[v_0]$)
 \ENDIF
 \IF[False branch]{edge $e_2 \in E_e$}
  \STATE RaiseV($v_2$, $L[v_0]$)
 \ENDIF
\ENDFOR
\ENDFOR
               \EMPTYLINE
\STATE \COMMENT{Obvious generalization applies for tests like ``$x\not=y$''}
\end{myalgorithmic}

\caption{A revised \code{Visit} procedure for SCC/SSI.}\label{alg:sccssi}
\end{myalgorithm}
Porting the SCC algorithm from SSA to SSI form 
(so that it takes information from conditionals into 
account) immediately increases
the number of constants we can find.
Only the \code{Visit} procedure must be updated for SCC/SSI: lattice
update rules for \sigfunction{s} must be added.
Algorithm~\ref{alg:sccssi} shows a new \code{Visit} procedure for the
two-level integer constant lattice of Wegman and Zadeck's SCC/SSA;
with this restricted value set only integer equality tests tap the
algorithm's full power.  The utility of SCC/SSI's \newterm{predicated
analysis} will become more evident as the value lattice is extended to
cover more constant types.
The time complexity of the updated algorithm is identical to that of
SCC/SSA: $O(E+U_{SSA})$.


\mycomment{
A simple example is shown in
Figure~\ref{fig:ssa_vs_ssi}: the version of the program on the right
is in SSI form, and SCC/SSI---unlike SCC/SSA---can
determine that \code{\subvar{foo}{2}} is a constant with value 1
(although nothing can be said about the value of
\code{\subvar{foo}{0}} or \code{\subvar{foo}{1}}) and therefore that
\code{\subvar{bar}{0}}, \code{\subvar{bar}{1}}, and
\code{\subvar{bar}{2}} are constants with the value 2.
SSI form creates a new name for \code{bar} at the conditional branch
to indicate that more information about its value is known.
}

\subsubsection{Extending the value domain}
\begin{myfigure}
\centering\renewcommand{\figscale}{0.5}\input{Figures/THlat2}
\caption{SCC value lattice extended to Java primitive value domain.}
\label{fig:scclat2}
\end{myfigure}
The first simple extension of the SCC value lattice enables us to
represent floating-point and other values.  For this work, we extended
the domain to cover the full type system of Java bytecode
\cite{gosling95:bytecode}; the extended lattice is presented in
Figure~\ref{fig:scclat2}.
The figure also introduces the abbreviated lattice notation we will use
through the following sections; it is understood that the
lattice entry labelled ``int'' stands for a finite-but-large set of
incomparable lattice elements, consisting (in this case) of the
members of the Java \code{int} integer type.
%\footnote{As the Java \code{int}
%type is a proper subset of the Java \code{long} integer type, we
%simplify our lattice by representing all integers as elements of
%\code{long}. Overflows are handled correctly after analysis.  Although
%the mechanics are slightly different, the same principle holds for
%the Java \code{double} and \code{float} types, and we similarly
%combine them in our lattice.}
Java \code{int}s are 32 bits long, so the ``int'' entry abbreviates
$2^{32}$ lattice elements.  Similarly, the ``double'' entry encodes not
the infinite domain of real numbers, but the domain spanned by the
Java \code{double} type which has fewer than $2^{64}$
members.\footnote{In IEEE-standard floating-point, some possible bit
patterns are not valid number encodings.}  The
Java \code{String} type is also included, to allow simple constant
string coalescing to be performed.  The propagation algorithm over
this lattice is a trivial modification to Algorithm~\ref{alg:sccssi}, and
will be omitted for brevity.  In the next sections, the ``int'' and ``long''
entries in this lattice will be summarized as ``Integer Constant'',
the ``float'' and ``double'' entries as ``Floating-point Constant'',
and the ``String'' entry as ``String Constant''.  As the lattice is
still only three levels deep, the asymptotic runtime complexity is
identical to that of the previous algorithm.

\subsubsection{Type analysis}
\begin{myfigure}[p]
\centering\renewcommand{\figscale}{0.5}\input{Figures/THlat3}
\caption{SCC value lattice extended with type information.}
\label{fig:scclat3}
\end{myfigure}
\begin{myfigure}[p]
\centering\renewcommand{\figscale}{0.33}\input{Figures/THlat4}
\caption{``Typed'' category of Figure~\ref{fig:scclat3} shown expanded.}
\label{fig:scclat4}
\end{myfigure}
\begin{myfigure}%
\begin{eqnarray*}
\code{int}\oplus\code{int}&=&\code{int}\\
\code{long}\oplus\{\code{int},\code{long}\}&=&\code{long}\\
\code{float}\oplus\{\code{int},\code{long},\code{float}\}&=&\code{float}\\
\code{double}\oplus\{\code{int},\code{long},\code{float},\code{double}\}&=&\code{double}\\
\code{String}\oplus\{\code{int},\code{long},\code{float},\code{double},\code{Object},\ldots\} &=& \code{String}
\end{eqnarray*}%
\caption{Java typing rules for binary operations.}
\label{fig:scc_typed_binop}
\end{myfigure}
\begin{mytable*}%[t]
\begin{tabular}{|l|l|c|c|c|}\hline
\small Hierarchy & \small Source language & \small Classes & \small Avg. depth & \small Max. depth \\ \hline
FLEX infrastructure & Java  &   550   &    1.9     &     5      \\
\code{javac} compiler & Java&   304   &    2.8     &     7      \\
NeXTStep 3.2$^\dag$& Objective-C & 488 &   3.5     &     8      \\
Objectworks 4.1$^\dag$&Smalltalk & 774 &   4.4     &    10      \\ \hline
\end{tabular}\\%
{\small$\dag$ indicates data obtained from Muthukrishnan and M\"uller
 \cite{muthukrishnan96:ch}.}
\caption{Class hierarchy statistics for several large O-O projects.}
\label{tab:chstats}
\end{mytable*}
\begin{myalgorithm}\small
Visit($n$:node) =
\begin{myalgorithmic}
\FOREACH{assignment ``$v \gets x \oplus y$'' in $n$}
 \STATE RaiseV($v$, $V[x] \oplus V[y]$)
 \COMMENT{binop rule: see figure~\ref{fig:scc_typed_binop}}
\ENDFOR
               \EMPTYLINE
\FOREACH{assignment ``$v \gets \text{MEM}(\ldots)$'' or 
                    ``$v \gets \text{CALL}(\ldots)$'' in $n$}
 \LET $t$ be the type of the \code{MEM} or \code{CALL} expression
 \STATE RaiseV($v$, $t$)
\ENDFOR
               \EMPTYLINE
\FOREACH{assignment ``$v \gets \phi(x_1,\ldots,x_n)$'' in $n$}
 \FOREACH{variable $x_i$ corresponding to predecessor edge $e_i$ of $n$}
  \IF{$e_i \in E_e$}
   \STATE RaiseV($v$, $\bigsqcup_L\{V[v],V[x_i]\}$)
   \COMMENT{meet rule: use least upper bound}
  \ENDIF
 \ENDFOR
\ENDFOR
               \EMPTYLINE
\FOREACH{branch ``if $x=y$ goto $e_1$ else $e_2$'' in $n$}
 \IF{$\text{Typed}\latleq L[x]$ or $\text{Typed}\latleq L[y]$}
  \STATE RaiseE($e_1$)
  \STATE RaiseE($e_2$)
 \ELSIF[if $x$ and $y$ are constants\ldots]{$L[x]=c$ and $L[y]=d$}
  \IF{$c=d$}
   \STATE RaiseE($e_1$)
  \ELSE
   \STATE RaiseE($e_2$)
  \ENDIF
 \ENDIF
\FOREACH{assignment ``$\tuple{v_1,v_2}\gets\sigma(v_0)$'' associated
         with this branch}
 \IF{edge $e_1 \in E_e$ and variable $v_0$ is the $x$ or $y$ in the test}
  \STATE\COMMENT{type error in source program if $L[x]$ and $L[y]$ are incomparable}
  \STATE RaiseV($v_1$, min($L[x]$, $L[y]$))
 \ELSIF{edge $e_1 \in E_e$}
  \STATE RaiseV($v_1$, $L[v_0]$)
 \ENDIF
 \IF[False branch]{edge $e_2 \in E_e$}
  \STATE RaiseV($v_2$, $L[v_0]$)
 \ENDIF
\ENDFOR
\ENDFOR
               \EMPTYLINE
\STATE \COMMENT{Obvious generalization applies for tests like ``$x\not=y$''}
\STATE \COMMENT{Obvious generalization applies for tests like ``$x
 \text{ instanceof } C$''}
\end{myalgorithmic}
\caption{\code{Visit} procedure for typed SCC/SSI.}
\label{alg:scctyped}
\end{myalgorithm}
In Figure~\ref{fig:scclat3} we extend the lattice to compute Java type
information.  The new lattice entry marked ``Typed'' is actually
forest-structured as shown in Figure~\ref{fig:scclat4}; it is as deep
as the class hierarchy, and the roots and leaves are all comparable to
$\top$ and $\bot$.  Only the \code{Visit} procedure must be modified;
the new procedure is given as Algorithm~\ref{alg:scctyped}.
Because the lattice $L$ is
deeper, the asymptotic runtime complexity is now $O(E+U_{SSA}D_c)$
where $D_c$ is the maximum depth of the class hierarchy.  
To form an estimate of the magnitude of $D_c$, Table~\ref{tab:chstats}
compares class hierarchy statistics for several large
object-oriented projects in various source languages. Our FLEX
compiler infrastructure, as a typical Java example, has an average
class depth of 1.91.\footnote{Measured August 2, 1999; the
infrastructure is under continuing development.}
In a forced example, of course, one can make the class depth $O(N)$;
however, one can infer from the data given that in real code the $D_c$
term is not likely to make the algorithm significantly non-linear.

A brief word on the roots of the hierarchy forest in Figure~%
\ref{fig:scclat4} is called for: Java has both a class hierarchy,
rooted at \code{java.lang.Object}, and several primitive types, which
we will also use as roots.  The primitive types include
\code{int}, \code{long}, \code{float}, and
\code{double}.\footnote{In the type system our infrastructure uses
(which is borrowed from Java bytecode) the \code{char},
\code{boolean}, \code{short} and \code{byte} types are folded into
\code{int}.}  Integer constants in the lattice are comparable to and
less than the \code{int} or \code{long} type; floating-point constants
are likewise comparable to and less than either \code{float} or
\code{double}.  String constants are comparable to and less than the
\code{java.lang.String} non-primitive class type.

The \code{void} type, which is the type of the expression \code{null},
is also a primitive type in Java; however we wish to keep $x \meet y$
identical to $\bigsqcup_L\{x, y\}$ (the least upper bound of $x$ and
$y$) while satisfying the Java typing rule that $\code{null} \meet x = x$
when $x$ is a non-primitive type and not a constant.  This requires
putting \code{void} comparable to but less than every non-primitive
leaf in the class hierarchy lattice.

The Java class hierarchy also includes \newterm{interfaces}, which are
the means by which Java implements multiple inheritance.  Base
interface classes (which do not extend other interfaces) are additional
roots in the hierarchy forest, although no examples of this are shown
in Figure~\ref{fig:scclat4}.

Since untypeable variables are generally forbidden, no operation
should ever raise a lattice value above ``Typed'' to $\top$.  The
otherwise-unnecessary $\top$ element is retained to indicate error
conditions.

This variant of the constant-propagation algorithm allows us to
eliminate unnecessary \code{instanceof} checks due to type-casting or
type-safety checks.  Section~\ref{sec:sptc_results} will provide
experimental validation of its utility.

Finally, note that the ability to represent \code{null} as the
\code{void} type in the lattice begins to allow us to address
null-pointer checks, although because $\code{null} \meet x =x$ for
non-primitive types we can only reason about variables which can be
proven to be null, not those which might be proven to be non-null
(which is the more useful case).  The next section will provide a more
satisfactory treatment.

\subsubsection{Array-bounds and null-pointer checks}
\begin{myfigure}
\centering\renewcommand{\figscale}{0.5}\input{Figures/THlat5}
\caption{Value lattice extended with array and null information.}
\label{fig:scclat5}
\end{myfigure}
\begin{myfigure}
\[\begin{array}{l}
\forall C \in \text{Class},\:
  C_{\text{non-null}} \latlt C_{\text{possibly-null}}\\
\forall C \in \text{Class}_{\text{non-null}},\:
  \bigsqcup_L \{\code{void},C\} \in \text{Class}_{\text{possibly-null}}\\
\forall C \in \text{Class}_{\text{possibly-null}},\:
  \code{void} \latlt C\\
\forall C \in \text{Class}_{\text{non-null}},\:
  \tuple{\code{void},C}\notin\:\latleq\\
\end{array}\]%
Let $A(C, n)$ be a
function to turn a lattice entry representing a non-null array class
type $C$ into the lattice entry representing a said array class with
known integer constant length $n$.  Then for any non-null array class
$C$ and integers $i$ and $j$,
\[\begin{array}{l}
A(C, i) \latlt C\\
\tuple{A(C, i), A(C, j)} \in\:\latleq \text{ if and only if } i=j\\
\end{array}\]%
\caption{Extended value lattice inequalities.}\label{fig:arraynull_rules}
\end{myfigure}
\begin{myalgorithm}\small
Visit($n$:node) =
\begin{myalgorithmic}
\STATE\COMMENT{Binop and \phifunction rules as in algorithm~\ref{alg:scctyped}}
               \EMPTYLINE
\FOREACH{assignment ``$v \gets \text{MEM}(\ldots)$'' or 
                    ``$v \gets \text{CALL}(\ldots)$'' in $n$}
 \LET $t\in\text{Class}_{\text{possibly-null}}\cup\text{Class}_{\text{primitive}}$
      be the type of the \code{MEM} or \code{CALL}% expression
 \STATE RaiseV($v$, $t$)
\ENDFOR
               \EMPTYLINE
\FOREACH{array creation expression ``$v \gets \text{new T}[x]$''}
 \IF{$L[x]$ is an integer constant}
  \STATE RaiseV($v$, $A(T, L[x])$)
 \ELSE
  \STATE RaiseV($v$, $T_{\text{non-null}}$)
 \ENDIF
\ENDFOR
               \EMPTYLINE
\FOREACH{array length assignment ``$v \gets \text{arraylength}(x)$''}
 \IF{$L[x]$ is an array of known constant length $n$}
  \STATE RaiseV($v$, $n$)
 \ELSE
  \STATE RaiseV($v$, \code{int})
 \ENDIF
\ENDFOR
               \EMPTYLINE
\STATE \COMMENT{Branch rules as in algorithm~\ref{alg:scctyped}, with
               the obvious extension to allow tests against
               \code{null} to lower a lattice value from
               $\text{Class}_{\text{possibly-null}}$ to
               $\text{Class}_{\text{non-null}}$.}
\end{myalgorithmic}
\caption{\code{Visit} procedure outline with array and null information.}
\label{alg:arraynull_scc}
\end{myalgorithm}
\begin{myfigure}
\begin{samplecode}
x = 5 + 6;\\
do \{\\
\>y = new int[x];\\
\>z = x-1;\\
\>if (0 <= z \&\& z < y.length)\\
\>\>y[z] = 0;\\
\>else\\
\>\>x--;\\
\} while (P);\\
\end{samplecode}
\caption{An example illustrating the power of combined analysis.}
\label{fig:combined}
\end{myfigure}
At this point, we can expand the value lattice once more to allow
elimination of unnecessary array-bounds and null-pointer checks, based
on our constant-propagation algorithm.  The new lattice is shown in
Figure~\ref{fig:scclat5}; we have split the ``Typed'' lattice entry to
enable the algorithm to distinguish between non-null and possibly-null
values,\footnote{Values which are always-null were discussed in the
previous section; they are identified as having primitive type \code{void}.}
and added a lattice level for arrays of known constant length.  
Some formal definition of the new value lattice can be found in
Figure~\ref{fig:arraynull_rules}; the meet rule is still the least upper
bound on the lattice.  Modifications to the \code{Visit} procedure are
outlined in Algorithm~\ref{alg:arraynull_scc}.
Notice that we exploit the pre-existing integer-constant propagation to
identify constant-length arrays, and that our integrated approach
allows one-pass optimization of the program in Figure~\ref{fig:combined}.

\begin{myfigure}\newcommand{\implicitcheck}[1]{\underline{#1}}
\begin{samplecode}
\implicitcheck{if (10 < 0)}\\
\>\implicitcheck{throw new NegativeArraySizeException();}\\
int[] A = new int[10]; \\
\implicitcheck{if (0 < 0 || 0 >= A.length)}\\
\>\implicitcheck{throw new ArrayIndexOutOfBoundsException();}\\
A[0] = 1;\\
for (int i=1; i < 10; i++) \{\\
\>\implicitcheck{if (i < 0 || i >= A.length)}\\
\>\>\implicitcheck{throw new ArrayIndexOutOfBoundsException();}\\
\>A[i] = 0;\\
\}\\
\end{samplecode}
\caption[Implicit bounds checks on Java array references.]
{Implicit bounds checks (underlined) on Java array references.}
\label{fig:induction}
\end{myfigure}
Note that the variable renaming performed by the SSI form at
control-flow splits is essential in allowing the algorithm to do 
null-pointer check elimination.  However, the lattice we are using
can remove bound checks from an expression $A[k]$ when $k$ is a
constant, but not when $k$ is an bounded induction variable.
In the example of Figure~\vref{fig:induction}, the first two implicit
checks are optimized away by this version of the algorithm, but the
loop-borne test is not. 

A typical array-bounds check (as shown in the example \vpageref{fig:induction})
verifies that the index $i$ of the array reference satisfies the
condition $0\leq i < n$, where $n$ is the length of the
array.\footnote{Languages in which array indices start at 1 can be
handled by slight modifications to the same techniques.} By
identifying integer constants as either positive, negative, or zero
the first half of the bounds check may be eliminated.  This requires a
simple extension of the integer constant portion of the lattice,
outlined in Figure~\vref{fig:scclat6}, with
negligible performance cost.  However, handling upper bounds
completely requires a symbolic analysis that is out of the current
scope of this work.  Future work will use induction variable analysis
and integrate an existing linear programming approach
\cite{RR00:PLDI00} to fully address array-bounds checks.
\begin{myfigure}[t]
\centering\renewcommand{\figscale}{0.6}\input{Figures/THlat6}
\caption[An integer lattice for signed integers.]
{An integer lattice for signed integers. A classification into
negative (M), positive (P), or zero (Z) is grafted onto the standard
flat integer constant domain.  The \code{(M-P)} entry is duplicated to
aid clarity.}
\label{fig:scclat6}
\end{myfigure}
