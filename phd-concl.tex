\chapter{Conclusion}\label{cha:concl}
\epigraphhead[70]{\epigraph{%
``Begin at the beginning,'' the King said, very gravely,
``and go on till you come to the end: then stop.''}
{Lewis Carroll, \textit{Alice's Adventures in Wonderland}}
}

In this thesis I have demonstrated that it is possible to implement an
efficient strongly atomic software transaction system, \apex.
I have shown that
nonblocking transactions can be used in applications beyond
synchronization, including fault tolerance and backtracking search.  I
have presented implementation details to address the practical
problems of building such a system.  I have argued the transactions
should not be subject to limits on size or duration. I have presented
both software and hardware implementations free of such restrictions.
Finally, since the low overhead of the \apex software system allows it to be
profitably combined with a hardware transaction system, I have shown
how this hybrid, \hyx, yields fast execution of short and small transactions
while allowing fallback to software for large or complicated
transactions.

There is no escape: parallel systems are in our future.
Programming them does not have to be as fraught as it is presently, however.
I believe that transactions provide a programmer-friendly model
of concurrency which eliminates many potential pitfalls of our
current locking-based methodologies.

In this thesis I have presented several designs for efficient
transaction systems that enable the
transactional programming model.  The \apex software-only system runs on
current hardware, and LTM and UTM indicate possible directions for future
hardware.  There are challenges and design decisions
remaining, however.  How should I/O be handled?  What are the proper semantics
for nested transactions?  What loophole mechanisms are necessary to
allow information about a transaction's progress to escape?

I believe hybrid systems such as \hyx offer the best answer to these
challenges.  They
combine the
inherent speed of hardware systems with the
flexibility of software, allowing novel solutions to be attempted
without requiring that design decisions be cast in silicon.  The
flag-based \apex software transaction system described in this thesis
imposes low overhead, allowing transactional
programming to get off the ground without hardware support in the near
term, while later supporting the development of new transactional models and
methodologies as part of a hybrid system.

Designing correct transaction systems is not easy, however.  In the
appendix you will find a Promela model of my software transaction
system.  Automated verification was essential when designing and
debugging the system, uncovering via exhaustive enumeration race
conditions much too subtle for me to discover by other means.  I
believe any credible transaction system must be buttressed by formal
verification.
