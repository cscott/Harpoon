% -*-latex-*- This is a LaTeX document.
% $Id: transact.tex,v 1.24 2003-11-09 20:19:59 cananian Exp $
\documentclass[11pt,notitlepage]{article}
\usepackage[section,plain]{algorithm} % algorithm environment
\usepackage{alltt} % alltt environment
\usepackage{amstext} % the \text command for math mode
\usepackage{comdef} % common definitions
\usepackage{pdffonts} % PDF-friendly fonts
\author{C.~Scott~Ananian}
\title{Software Atomic Transactions in FLEX}
\date{\today \\ $ $Revision: 1.24 $ $}

% this is handy when talking about structure member references.
\newcommand{\ptr}{\code{->}}
% how big should code listings be?
\newcommand{\codesize}{\scriptsize}

\begin{document}
\bibliographystyle{abbrv}
\maketitle
\section{Data Structures}

Transaction support uses the same ``inflated object'' extension
mechanism which is used for thread synchronization, clustered heaps,
JNI data, and finalization for certain garbage collectors.
The basic object data structure, as shown in figure \ref{fig:oobj}, is
unchanged, although fields containing the specified
\texttt{FLAG\_VALUE} have different semantics.  The flag value is used
to indicate that the field value is ``not here''; that is, the code
must consult the transaction information to find the field's current
value.  This happens very rarely even when no transaction is
associated with the object; Shasta \cite{scales96:shasta} has shown that the
overhead entailed by such ``false'' transactions is expected to be
extremely low.

The ``inflated object'' data structure, shown in figure
\ref{fig:infl}, has an ``object versions'' linked list associated with
it.  Each entry conceptually stores information about a different,
possibly-uncommitted, version of the object.  One of these versions is
guaranteed to be ``most recent committed'', and it is guaranteed that
the ``most recent committed'' version will be the first committed
version in the list.

The commit record structure is shown in figure \ref{fig:commitrec}.
The pointer to the commit record also serves as a transaction
identifier.  Transactions can be nested, therefore every commit record
may contain a pointer to a ``parent'' transaction which it is
dependent upon.  The commit record also contains a three-value
``state'' field which indicates whether the transaction it represents
has been committed or aborted yet.

The ``object version'' structure, shown in figure
\ref{fig:version}, is similar to the object structure shown in
figure \ref{fig:oobj} so that field pointers can be converted easily.
Each \texttt{vinfo} structure identifies an object version which is
associated with a particular transaction (via \code{transid}) which
has mutated the object.  Readers are associated with each mutated
version using the \texttt{readers} list, which is inlined to avoid an
extra memory dereference in the hot path through \code{ReadTrans}
(figure \ref{fig:readtrans}).

Subtransactions cause a possible race condition when traversing the
versions list.  The parent of each subtransaction will have a separate
``pre-subtransaction'' version entry in order to allow retrying an
aborting subtransaction.  A non-transactional reader may examine the
subtransaction and find that it is still waiting and move on to the
next item in the versions list, which will be the subtransaction's
parent.  But if both the subtransaction and then its parent are then
committed before the non-transactional reader proceeds, it may
erroneously fetch the ``pre-subtransaction'' version of the
now-committed parent instead of the correct version associated with
the newly-committed subtransaction.

To prevent this race, there are two ``next'' fields for the versions
list.  The \code{anext} field is followed if the version's transaction
has been aborted.  This allows correct fail back to the parent
transaction's version in this case.  However, if the version's
transaction is in the \code{WAITING} state (review figure
\ref{fig:commitrec}), we skip examining its parent transaction(s),
avoiding the possible race, by following the \code{wnext} field.

\begin{figure}
\begin{alltt}\codesize
/* the oobj structure tells you what's inside the object layout. */
struct oobj \{
  struct claz *claz;
  /* if low bit is one, then this is a fair-dinkum hashcode. else, it's a
   * pointer to a struct inflated_oobj. this pointer needs to be freed
   * when the object is garbage collected, which is done w/ a finalizer. */
  union \{ ptroff_t hashcode; struct inflated_oobj *inflated; \} hashunion;
\textbf{#ifdef WITH_TRANSACTIONS
#define FLAG_VALUE 0xCACACACA
  /* consult transaction to determine value of any field with FLAG_VALUE */
#endif}
  /* fields below this point */
  char field_start[0];
\};
\end{alltt}
\caption{The object structure in the FLEX runtime.}
\label{fig:oobj}
\end{figure}
\begin{figure}
\begin{alltt}\codesize
/* the inflated_oobj structure has various bits of information that we
 * want to associate with *some* (not all) objects. */
struct inflated_oobj \{
  ptroff_t hashcode; /* the real hashcode, since we've booted it */
  void *jni_data; /* information associated with this object by the JNI */
  void (*jni_cleanup_func)(void *jni_data);
\textbf{  int use_count; /* determines when we can deflate the object */
  /* TRANSACTION SUPPORT */
#if WITH_TRANSACTIONS
  struct vinfo *first_version; /* linked list of object versions */
#endif}
  /* locking information */
#if WITH_HEAVY_THREADS || WITH_PTH_THREADS
  pthread_t tid; /* can be zero, if no one has this lock */
  jint nesting_depth; /* recursive lock nesting depth */
  pthread_mutex_t mutex; /* simple (not recursive) lock */
  pthread_cond_t  cond; /* condition variable */
  pthread_rwlock_t jni_data_lock; /*read/write lock for jni_data field, above*/
#endif
#ifdef WITH_CLUSTERED_HEAPS
  struct clustered_heap * heap;
  void (*heap_release)(struct clustered_heap *);
#endif
#ifdef BDW_CONSERVATIVE_GC
  /* for cleanup via finalization */
  GC_finalization_proc old_finalizer;
  GC_PTR old_client_data;
#endif
\};
\end{alltt}
\caption{The ``inflated object'' structure in the FLEX runtime.
         Fields specific to transaction support are in boldface.}
\label{fig:infl}
\end{figure}

\begin{figure}
\begin{alltt}\codesize
#if WITH_TRANSACTIONS
/* A simple linked list of transaction identifiers. */
struct tlist \{
  struct commitrec *transid; /* transaction id */ 
  struct tlist *next; /* next version. */
\};

/* The vinfo structure provides values for a given version of the object. */
struct vinfo \{
  struct commitrec *transid; /* transaction id */ 
  struct tlist readers; /* list of readers.  first node is inlined. */
  struct vinfo *anext; /* next version to look at if transid is aborted. */
  struct vinfo *wnext; /* next version to look at if transid is waiting. */
  /* fields below this point */
  char field_start[0];
\};
#endif
\end{alltt}
\caption{The ``version'' structure in the FLEX runtime.}
\label{fig:version}
\end{figure}

\begin{figure}
\begin{alltt}\codesize
#if WITH_TRANSACTIONS
/* The commit record for a (possibly nested) transaction */
struct commitrec \{
  /* The transaction that this depends upon, if any. */
  struct commitrec *parent;
  /* The 'state' should be initialized to W and write-once to C or A. */
  enum \{ WAITING=0, COMMITTED, ABORTED \} state;
\};
#endif
\end{alltt}
\caption{Commit record structure.}
\label{fig:commitrec}
\end{figure}

These data structures allow a single-writer multiple-readers
transaction consistency scheme to be implemented.  To avoid polling
where possible, our transactions prefer committing suicide to aborting
another transaction when consistency only allows a single outstanding
transaction.  At the end of every versions list we guarantee there is
at least one committed transaction.  Other ordering is also preserved
in the list: for every \code{vinfo} structure $v$,
$v\ptr\code{transid}$ is a subtransaction of
$v\ptr\code{anext}\ptr\code{transid}$ and \emph{not} a subtransaction of
$v\ptr\code{wnext}\ptr\code{transid}$.  Every reader is a
subtransaction of every writer; formally, for every reader $r$ in
$v\ptr\code{readers}$, $r\ptr\code{transid}$ is a subtransaction of
$v\ptr\code{transid}$.  Finally, to enforce consistency, there are
never any non-aborted readers of $v$'s parent transaction which are
subtransactions of $v$; that is, for all readers $r$ in
$v\ptr\code{anext}\code{readers}$, $v\ptr\code{transid}$ is a
subtransaction of $r\ptr\code{transid}$ (and not the other way
around), unless $r\ptr\code{transid}$ has been aborted.

\section{Algorithms}

Blah blah blah

\begin{figure}
$\text{Read}(\code{struct oobj *} o, \code{int } n)$ =
\begin{myalgorithmic}
\STATE $x \gets o\ptr \code{field}[n]$
\IF{$x = \code{FLAG\_VALUE}$}

\STATE $v \gets o \ptr \code{hashunion.inflated} \ptr \code{first\_version}$
\STATE $u \gets 0$ \COMMENT{count uncommitted transactions}
\LOOP
  \ASSERT $v \neq \code{null}$
  \MATCHING{$\text{StateP}''(v)$}
   \CASE{\code{COMMITTED}}
    \ASSERT \small$\forall n,\: o \ptr \code{field}[n]=\code{FLAG\_VALUE}
             \Rightarrow  v \ptr \code{field}[n] \text{ is valid }$
    \BREAK \textbf{loop}
   \ENDCASE
   \CASE{\code{ABORTED}}
    \STATE $v \gets v\ptr\code{anext}$
   \ENDCASE
   \CASE{\code{WAITING}}
    \STATE $u \gets u + 1$
    \STATE $v \gets v \ptr \code{wnext}$
   \ENDCASE
  \ENDMATCHING
\ENDLOOP*
\STATE $x \gets v \ptr \code{field}[n]$
\IF{$u=0 \wedge x \neq \code{FLAG\_VALUE}$}
%  \STATE \COMMENT{XXX: race here}
  \STATE $o \ptr \code{field}[n] \gets x$ \COMMENT{store must be atomic}
  \STATE $\code{dec\_use}(o)$
         \COMMENT{will deflate when all fields are copied back}
\ENDIF
\ENDIF
\RETURN $x$
\end{myalgorithmic}
\caption{Algorithm for non-transactional read of a field.}
\label{fig:readnotrans}
\end{figure}

\begin{figure}
$\text{Write}(\code{struct oobj *} o, \code{int } n,
              \code{<ty> } x)$ =
\begin{myalgorithmic}
\IF{$o\ptr\code{field}[n] = \code{FLAG\_VALUE}$}\label{line:wL1}
  \IF[try to avoid aborting anything]{$\text{PossiblyDownGrade}(o, n)$}
    \GOTO line \ref{line:wL1}
  \ENDIF
  \STATE $v \gets o \ptr \code{hashunion.inflated} \ptr \code{first\_version}$
  \WHILE{$\text{AbortTransaction}(v\ptr\code{transid}) \neq \code{COMMITTED}$}
    \STATE \COMMENT{prune list for the sake of writes w/o intervening reads}
%    \STATE \COMMENT{XXX: potential race here --- v could be already removed}
    \STATE $v \gets v\ptr\code{anext}$
    \STATE $o \ptr \code{hashunion.inflated} \ptr \code{first\_version}
            \gets v$
  \ENDWHILE
  \STATE \COMMENT{v is a committed transaction}
  \STATE $v\ptr\code{anext} \gets v\ptr\code{wnext} \gets \code{null}$
         \COMMENT{atomic stores}
  \STATE \COMMENT{abort all readers of this committed transaction}
  \STATE $r \gets \code{\&}(v\ptr\code{readers})$
  \WHILE{$r \neq \code{null}$}
    \STATE $\text{AbortTransaction}(r\ptr\code{transid})$
    \STATE $r \gets r\ptr\code{next}$
  \ENDWHILE
\ELSIF{$x = \code{FLAG\_VALUE}$}
%  \STATE \COMMENT{XXX: define CreateNewVersion (watch those races)}
% when we define CreateNewVersion, we'll need: o\ptr\code{claz}\ptr\code{size}
  \STATE $v \gets \text{CreateNewVersion}(o, \code{null}, o)$
\ELSE
  \STATE $v \gets o$
\ENDIF
\STATE $v\ptr\code{field}[n] \gets x$
\end{myalgorithmic}
\caption{Algorithm for non-transactional write of a field.}
\label{fig:writenotrans}
\end{figure}

\begin{figure}
$\text{ReadTrans}(\code{struct commitrec *}c, 
                  \code{struct oobj *} o, \code{int } n)$~=
\begin{myalgorithmic}
\STATE $\code{InflateObject}(o)$
\STATE $v \gets o \ptr \code{hashunion.inflated} \ptr \code{first\_version}$
\IF{$v = \code{null}$}
  \STATE $v \gets \text{CreateNewVersion}(o, \code{null}, o)$
\ENDIF
\STATE $r \gets \code{\&}(v\ptr\code{readers})$\label{line:rtL2}

\REPEAT
  \IF{$r\ptr\code{transid} = c$}
    \GOTO line \ref{line:rtL1}
  \ENDIF
  \STATE $r \gets r\ptr\code{next}$
\UNTIL{$r = \code{null}$}

\MATCHING{$\text{StateP}''(v)$}
  \CASE{\code{ABORTED}}
    \STATE $v \gets v\ptr\code{anext}$
    \GOTO line \ref{line:rtL2}
  \ENDCASE
  \CASE{\code{COMMITTED}}
    \GOTO line \ref{line:rtL3}
  \ENDCASE
  \CASE{\code{WAITING}}
    \STATE\COMMENT{abort unless $c$ is a non-aborted subtransaction
                   of $v\ptr\code{transid}$}
    \IFNOT{$\text{IsNAST}(c, v)$}
      \IF{$\text{PossiblyDownGrade}(o, n)$}
        \GOTO line \ref{line:rtL2} \COMMENT{last chance for life}
      \ELSE
        \STATE commit suicide \COMMENT{can't guarantee consistency}
      \ENDIF
    \ENDIF
    \GOTO line \ref{line:rtL3}
  \ENDCASE
\ENDMATCHING
\STATE add $c$ to head of readers list, atomically.\label{line:rtL3}

\STATE \COMMENT{$v$ has correct version} \label{line:rtL1}

\IF{$o\ptr\code{field}[n] \neq \code{FLAG\_VALUE}$}
%  \STATE \COMMENT{XXX: race here, most likely}
  \STATE copy current value somewhere (how far over?)
  \STATE $\code{inc\_use}(o)$
  \STATE $o\ptr\code{field}[n] \gets \code{FLAG\_VALUE}$
%  \STATE \COMMENT{XXX: even if this transaction is aborted now, we
%                  still register a dependency on this field.}
\ENDIF
\RETURN $v\ptr\code{field}[n]$
\end{myalgorithmic}
\caption{Algorithm for transactional read of a field.}
\label{fig:readtrans}
\end{figure}

\begin{figure}
$\text{WriteTrans}(\code{struct commitrec *}c, 
                   \code{struct oobj *} o, \code{int } n,
                   \code{<ty> } x)$~=
\begin{myalgorithmic}
\STATE $\code{InflateObject}(o)$
\STATE $v \gets o \ptr \code{hashunion.inflated} \ptr \code{first\_version}$
\IF{$v = \code{null}$}
  \STATE $v \gets \text{CreateNewVersion}(o, c, o)$
\ELSIF{$v\ptr\code{transid} \neq c$}\label{line:wtL1}
  \MATCHING{$\text{StateP}''(v)$}
    \CASE{\code{ABORTED}}
      \STATE $v \gets v\ptr\code{anext}$
      \GOTO line \ref{line:wtL1}
    \ENDCASE
    \CASE{\code{COMMITTED}}
      \GOTO line \ref{line:wtL2}
    \ENDCASE
    \CASE{\code{WAITING}}
      \STATE\COMMENT{abort unless $c$ is a non-aborted subtransaction
                     of $v\ptr\code{transid}$}
      \IFNOT{$\text{IsNAST}(c, v)$}
        \IF{$\text{PossiblyDownGrade}(o, n)$}
          \GOTO line \ref{line:rtL2} \COMMENT{last chance for life}
        \ELSE
          \STATE commit suicide \COMMENT{can't guarantee consistency}
        \ENDIF
      \ENDIF
      \GOTO line \ref{line:wtL2}
    \ENDCASE
  \ENDMATCHING
  \STATE kill some of the readers\label{line:wtL2}
  \STATE $v \gets \text{CreateNewVersion}(o, c, v)$
\ENDIF
\STATE \COMMENT{$v$ has correct version}

\IF{$o\ptr\code{field}[n] \neq \code{FLAG\_VALUE}$}
  \STATE watch for races here
  \STATE copy current value somewhere (how far over?)
  \STATE $\code{inc\_use}(o)$
  \STATE $o\ptr\code{field}[n] \gets \code{FLAG\_VALUE}$
\ENDIF
\STATE $v\ptr\code{field}[n] \gets x$
\end{myalgorithmic}
\caption{Algorithm for transactional write of a field.}
\label{fig:writetrans}
\end{figure}

\begin{figure}
$\text{StateP}(\code{struct commitrec *} c)$ =
\begin{myalgorithmic}
\LOOP
  \IF{$c = \code{null}$}
    \RETURN \code{COMMITTED}
  \ENDIF
  \STATE $s \gets c \ptr\code{state}$
         \COMMENT{cache this read to prevent races}
  \IF{$s \neq \code{COMMITTED}$}
    \RETURN $s$
  \ENDIF
  \STATE $c \gets c \ptr \code{parent}$
\ENDLOOP*
\end{myalgorithmic}
\caption{Simple algorithm for determining transaction status.}
\label{fig:statep}
\end{figure}

\begin{figure}
$\text{StateP}'(\code{struct commitrec **} c_p)$ =
\begin{myalgorithmic}
\STATE $c \gets \code{*}c_p$
\STATE $s \gets \code{COMMITTED}$
\STATE $l \gets \code{false}$
\WHILE{$c \neq \code{null}$}
  \STATE $s \gets c\ptr\code{state}$ \COMMENT{atomic}
  \IF{$s \neq \code{COMMITTED}$}
    \BREAK
  \ENDIF
  \STATE $c \gets c \ptr \code{parent}$
  \STATE $l \gets \code{true}$
\ENDWHILE
\IF[$l$ indicates a change has been made]{$l$}
  \STATE $\code{*}c_p \gets c$ \COMMENT{atomic}
\ENDIF
\RETURN $\tuple{s,l}$
\end{myalgorithmic}
\caption{Improved algorithm for determining transaction status ---
         does pruning.}
\label{fig:statepprime}
\end{figure}

\begin{figure}
$\text{StateP}''(\code{struct vinfo *} v)$ =
\begin{myalgorithmic}
\MATCHING{$\text{StateP}'(\code{\&}(v\ptr\code{transid}))$}
  \CASE{\tuple{\code{ABORTED}, l}}
%    \STATE \COMMENT{XXX: potential race here --- v could be already removed}
    \STATE \COMMENT{the following store ought to be atomic}
    \STATE $o \ptr \code{hashunion.inflated} \ptr \code{first\_version}
            \gets v\ptr\code{anext}$
    \RETURN \code{ABORTED}
  \ENDCASE
  \CASE{\tuple{\code{COMMITTED}, l}}
    \STATE \COMMENT{these stores should be atomic (but no one will read them)}
    \STATE \COMMENT{this might be better done by the gc}
    \STATE $v\ptr\code{anext} \gets v\ptr\code{wnext} \gets \code{null}$
    \RETURN \code{COMMITTED}
  \ENDCASE
  \CASE{\tuple{\code{WAITING}, l}}
    \IF[transid was pruned; snip out parents]{$l$}
      \WHILE{$v\ptr\code{transid} = v\ptr\code{anext}\ptr\code{transid}$}
        \STATE $v\ptr\code{anext} \gets v\ptr\code{anext}\ptr\code{anext}$
%               \COMMENT{XXX: race here.}
      \ENDWHILE
    \ENDIF
    \RETURN \code{WAITING}
  \ENDCASE
\ENDMATCHING
\end{myalgorithmic}
\caption{Even better pruning algorithm for determining transaction status.}
\label{fig:statepprimeprime}
\end{figure}

\begin{figure}
$\text{AbortTransaction}(\code{struct commitrec *} c)$ =
\begin{myalgorithmic}
\IF{$c = \code{null}$}
  \RETURN \code{COMMITTED}
\ENDIF
\LOOP
  \STATE $s \gets c\ptr\code{state}$
  \IF{$s \neq \code{WAITING}$}
    \RETURN $s$
  \ENDIF
  \STATE $\code{compare\_and\_swap}(\&c\ptr\code{state}, s, \code{ABORTED})$
         \label{line:abort}
\ENDLOOP*
\end{myalgorithmic}
\caption{Algorithm for (possibly) aborting a transaction.
         Change the constant \code{ABORTED} in line \ref{line:abort}
         to \code{COMMITTED} to obtain the \code{CommitTransaction} algorithm.}
\label{fig:aborttrans}
\end{figure}

\begin{figure}
$\text{IsNAST}(\code{struct commitrec *}c_s,\code{struct vinfo *}v)$ =
\begin{myalgorithmic}
\STATE $c_p \gets v\ptr\code{transid}$
\IF{$c_s \neq c_p$}
  \IF{$c_p \neq \code{null} \wedge c_p\ptr\code{state} = \code{COMMITTED}$}
     \label{line:nast0}
    \REPEAT
      \STATE $c_p \gets c_p\ptr\code{parent}$
    \UNTIL{$c_p = \code{null} \vee c_p\ptr\code{state} \neq \code{COMMITTED}$}
          \label{line:nast1}
    \STATE $v\ptr\code{transid} \gets c_p$ \COMMENT{atomic}
    \STATE check to see whether successor needs to be pruned.
  \ENDIF
  \IF[committed all the way down]{$c_p = \code{null}$}
    \STATE $v\ptr\code{anext} \gets v\ptr\code{wnext} \gets \code{null}$
           \COMMENT{prune}
  \ELSIF{$c_p\ptr\code{state} = \code{ABORTED}$}\label{line:nast2}
    \RETURN \code{false} \COMMENT{$c_s$ either not subtrans, or is aborted.}
  \ELSE
    \STATE\COMMENT{the possible WAITING-becomes-COMMITTED race}
    \STATE\COMMENT{between l.\ref{line:nast0}/l.\ref{line:nast1} and l.\ref{line:nast2} is not harmful.}
    \WHILE{$c_s \neq c_p$}
      \IF{$c_s\ptr\code{state} = \code{ABORTED}$}
        \RETURN \code{false} \COMMENT{aborted}
      \ENDIF
      \STATE $c_s \gets c_s\ptr\code{parent}$
      \IF{$c_s = \code{null}$}
        \RETURN \code{false} \COMMENT{not a subtransaction}
      \ENDIF
    \ENDWHILE
  \ENDIF
\ENDIF
\RETURN $(\text{StateP}(c_s) \neq \code{ABORTED})$
\end{myalgorithmic}
\caption{Algorithm for determining whether $c_s$ is a non-aborted
         subtransaction of $v\ptr\code{transid}$}
\label{fig:isnast}
\end{figure}

%% XXX: separate transactions?  parallel subtransactions?
%% XXX: can I have two reads outstanding on an object? (yes?)

%\begin{figure}
%\begin{alltt}\codesize
%\end{alltt}
%\caption{}
%\label{fig:}
%\end{figure}

\bibliography{harpoon}
\end{document}
